% TODO remove draft to get syntax highlighting in code
\documentclass[12pt,a4paper]{report}
\usepackage[margin=110pt]{geometry}
\usepackage{hyperref}
\usepackage[parfill]{parskip}
\usepackage{minted}
\usepackage{subfiles}
\usepackage{multicol}
\usepackage{color}
\usepackage[dvipsnames]{xcolor}
\definecolor{dgreen}{rgb}{0.,0.6,0.}

% Set minted size
%\usepackage{etoolbox}
%\AtBeginEnvironment{minted}{\fontsize{11.75}{11.75}}

% This is used to specify fonts and configurationss
\usepackage[no-math]{fontspec}
% Sets a monofont with unicode character
\setmonofont[Scale=MatchLowercase]{Apple Symbols}


% no red boxes on parser error:
\makeatletter
\expandafter\def\csname PYGdefault@tok@err\endcsname{\def\PYGdefault@bc##1{{\strut ##1}}}
\makeatother

\usepackage{bussproofs}
\usepackage{amsmath}
\usepackage{amssymb}
\usepackage{pifont}% http://ctan.org/pkg/pifont

% TODO notes
\usepackage[textwidth=3.7cm]{todonotes} \setlength{\marginparwidth}{3.7cm}

% Alternative Font + fill for some missing chars
%\setmonofont[Scale=MatchLowercase]{Menlo}
%\usepackage{MnSymbol}
\usepackage{newunicodechar}

\newcommand{\cmark}{\ding{51}}%
\newcommand{\xmark}{\ding{55}}%

%\newunicodechar{⊥}{$\bot$}
%\newunicodechar{⟪}{$\llangle$}
%\newunicodechar{⟫}{$\rrangle$}
\newunicodechar{ᶜ}{$^c$}
\newunicodechar{ₑ}{$_e$}
\newunicodechar{ⱽ}{$^v$}
\newunicodechar{ˢ}{$^s$}
\newunicodechar{ᵗ}{$^t$}
\newunicodechar{✗}{\xmark}

\usepackage{pdfpages}

\begin{document}

\includepdf[pages={1}]{frontpage.pdf}

%
%\begin{titlepage}
%\begin{center}
%
%        \vspace*{1cm}
%
%\textbf{SVC \\}
%\textbf{A prototype of a Structure-aware Version Control system}
%
%        \vspace{1.5cm}
%
%\end{center}
%
%\vfill
%
%\textbf{Marco Vassena \\ ICA-4110161} 

%Computer Science \\
%Utrecht University \\
%The Netherlands\\
%\date{\today}

\abstract{
This thesis studies the problem of structure-aware revision control,
which consists of exploiting the knowledge of the structure
of data to improve the quality of version control systems.
Formats are firstly described using an EDSL, which distinguishes 
meta-data from the actual content. From the unique format specification 
inverse-by-construction parser and printer are derived.
The data stored in a file is converted into a
heterogeneous rose tree, a generic representation of algebraic data types,
used by a \texttt{diff} and \texttt{diff3} algorithm to respectively 
detect changes and merge revisions.
Lastly the semantics and the properties of the two algorithms are studied 
with a formal model developed in the Agda proof assistant. 
}

%The edit scripts generated are more precise and consider only, thus avoiding unnecessary 
%conflicts.

\tableofcontents

	
\subfile{./Introduction/Introduction}
\subfile{./Format/Format.tex}
\subfile{./FormalModel/FormalModel.tex}
 
\graphicspath{{Implementation/}}
\subfile{./Implementation/Implementation.tex}

\subfile{./Example/Example.tex}
\subfile{./Conclusion/Conclusion.tex}

\newpage

\section*{Acknowledgments}
I would like to thank my supervisor Dr. Wouter Swiestra for his guidance
and constant support throughout this project. 
Whenever I was stuck, he could set me on the right track to a solution,
clarifying my problems and putting them in right words.
My gratitude goes also to the Software Technology group, whose
professors have inspired me and taught me so much in these two years.
Their enthusiasm has sparkled in me the interest for functional programming and programming languages technologies to the point that I decided
to pursue a PhD in these topics.
I also gladly remember the afternoons spent with the Software Technology reading club, during which we were busy getting the most out of interesting papers and in constructive discussions.

I would like to thank from the bottom of my heart my parents for 
supporting me throughout my studies. None of this would have been possible
without you. Studying at Utrecht University is a great opportunity and
my only regret is that it would not let me see you, my sister Chiara, my hopefully future brother-in-law Mattia and my grandmother Clara as often as I wished.
A special thanks goes to Laura for putting up with me in the busiest moments
and sharing with me this fantastic experience.
There a	re so many great adventures we have lived in these two years and
thanks to you there is plenty of pictures to remember them all.
I would like to thank my friends from Utrecht: João Pizani, Philipp, Wout and João Alpuim. I really had a great time with all of you and I hope we will
keep in touch in the next years.
I would like to thank also my friends from Delft: Enrico, Gherardo, Marco,
Oana and Bronius. You are wonderful friends and I am very lucky to
have met you.

Lastly a special thanks goes to the Haskell Cafe and Agda mailing list
for their concrete help, without which I could have not completed
this project.
These communities have provided me with very interesting insights and answered to my very technical questions.

\bibliographystyle{plain}
\bibliography{svc}

\end{document}
