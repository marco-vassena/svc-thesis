\documentclass[../Thesis.tex]{subfiles}

\begin{document}

\chapter{Haskell Implementation}
This chapter presents the Haskell implementation of the \texttt{diff}
and \texttt{diff₃} algorithms discussed in \ref{chapter:FormalModel}.

	\subsection*{Motivation}
	Even though Agda's type system is sophisticated, it is still a research 
	prototype and its compiler is still rather immature and does not produce
	efficient code. Consequently a more mature programming language has been 
	chosen, in order to provide a practical implementation of
	the algorithms studied in this thesis.
	Haskell, a general-purpose, strongly typed, purely functional 
	programming language \cite{Hudak07ahistory, Marlow_haskell2010} 
	has been used for this purpose.
	Even though full-fledged dependent types are not available, the 
	Glasgow Haskell Compiler (GHC) \cite{GHC}, the current state-of-the-art, 
	optimizing compiler for Haskell, provides several extensions to the type
	system, which allow to partially simulate dependently typed programming.
	\todo{Mention which version GHC was used}
	The implementation discussed in this chapter 
	relies heavily on Generalized Algebraic Data Types 
	(GADT) \cite{JonWasWei04,PeytonJonesSUT,SchGADT} and Type Families
	\cite{ChakATC,ChakATS,SchrTCO,SchrTOT,Kiselyov09funwith}.
	In the following their limitations with respect to the formal model will be 
	remarked.
	Furthermore Type Classes provide a convenient way to make the 
	implementation modular.

	\todo{Different font for Type Class, Type Family and GADT?}
	
	\section{Basics}
	This section presents the basic data types and type classes used 
	in the algorithms section.

	\subsection{Type Manipulation}

	\paragraph{Proxy}
	Programs that inspect and manipulate types need specific data types
	and functions, whose main role is to fix types and 
	ultimately satisfy the type checker, rather than computing values.
	For example the data type \texttt{Proxy a} is a poly-kinded proxy type that 
	has only one non-bottom value, namely \texttt{Proxy}, and it
	is usually used to specify types in signatures.
	
\begin{minted}	{haskell}
data Proxy t = Proxy
\end{minted}

	This data type will be used quite often in the rest of the chapter.
	
	\paragraph{Type Equality}
	In the formal model a decidable type equality binary operator was 
	employed in the \texttt{diff} algorithm and to type check
	merged edit script. This kind of function is usually not available 
	in strongly typed	languages like Haskell, because type are erased 
	at run time. However, when the set of types involved is limited to a
	finite set known at compile time, it is possible to implement such
	an operator for them.

	The type \texttt{Elem a xs} denotes that the type \texttt{a}
	belongs to the list of types\texttt{xs}.
	Its definition closely resembles the membership 
	data types discussed in the formal model:
	
\begin{minted}{haskell}
data Elem a as where
  Here :: Elem a (a ': as)
  There :: Elem a as -> Elem a (b ': as)
\end{minted}

	Assuming that \texttt{as} does not contain duplicates it is possible
	to test type equality for types that belong to it:
\begin{minted}{haskell}
cmpTy :: Elem a as -> Elem b as -> Maybe (a :~: b)
cmpTy Here Here = Just Refl
cmpTy (There i) (There j) = cmp i j
cmpTy _ _ = Nothing
\end{minted}	

	The two arguments act as indexes in the \emph{same}
	list \texttt{as}. If they happen to point to the same element in the
	list, then equality follows, otherwise the two types are different.

	\subsection{Universe}
	\label{subsec:Universe}
	In the formal model the presence of a certain type 
	\texttt{F : List Set -> Set -> Set} and some operations to manipulate
	them were assumed using a postulate. In the actual implementation
	a concrete type with this kind must be defined for the family
	of mutually recursive data types at hand. Nevertheless, 
	Haskell's type class system support higher kinds, hence it is easy to
	abstract over them in the implementation using a type class.
	
\begin{minted}{haskell}
class Family f where
  type TypesOf f :: [ * ]
  (=?=) :: f as a -> f bs b -> Maybe (a :~: b , as :~: bs) 
  argsTy :: f as a -> TList f as
  outputTy :: f as a -> TList f '[ a ]
  string :: f as a -> String
\end{minted}

	The type parameter \texttt{f} has kind \texttt{[ * ] -> * -> *}, in
	Haskell kind signatures, where \texttt{*} corresponds to \texttt{Set} in
	Agda and \texttt{[ * ]} to \texttt{List Set}. 
	A value of type \texttt{f xs a} represents a concrete constructor
	of a data type of type \texttt{a}, that takes arguments of types determined 
	by \texttt{as}.
	The associated type \texttt{TypesOf f} is a list containing all the types
	that appear in the family of mutually recursive data types embodied by
	\texttt{f}. The only important requirement needed to ensure consistency
	is that this list represents a set, therefore with no duplicates. Furthermore
	the order in which the types are listed is irrelevant.
	The first method \texttt{=?=} tests for equality between nodes, analogously
	to the operator described in the formal model, except with a less 
	refined signature.
	The methods \texttt{argsTy} and \texttt{outputTy} reify respectively the
	arguments and output types of a constructor and are needed for
	type checking merged edit scripts. This topic
	will be explained in more details in the following.
	Lastly the method \texttt{string} returns a string representation of a
	constructor and it is used exclusively for to interact with the user.
	
	The second type class defined relates single types and a family:
\begin{minted}{haskell}
class (Family f, ShowType a) => a :<: f where
  getElem :: Proxy f -> Elem a (TypesOf f)
  toDTree :: a -> DTree f a 
  fromDTree :: DTree f a -> a
\end{minted}
	
	The constraints \texttt{a :<: f} denotes that the type \texttt{a} is part
	of the family \texttt{f}.
	As such the first method ensures that \texttt{a} is among the types
	involved in the family \texttt{f}.
	The methods \texttt{toDTree} and \texttt{fromDTree} are used to convert
	the user defined data types to the well-typed generic tree representation
	and the other way around.
	The class \texttt{ShowType} provides the method \texttt{showTy}, which
	returns a string representation of the type \texttt{a} and it is
	used exclusively for error reporting when type checking. 

\begin{minted}{haskell}
class ShowType a where
  showTy :: Proxy a -> String
\end{minted}
	
	\subsection{Typed List}
	The algorithms manipulate several different kind of typed list,
	that are all small variation of the heterogeneous list \texttt{HList} 
	introduced in \ref{chapter:FormatRepresentation} \todo{Specific section 
	about HList}. This section defines them and explains their role.
	
	\paragraph{TList}
	The type \texttt{TList f as} represents a list of types \texttt{as} each 
	of them belonging to the family \texttt{f}.

\begin{minted}{haskell}
data TList f as where
  TNil :: TList f '[]
  TCons :: (a :<: f) => Proxy a -> TList f as -> TList f (a ': as)
\end{minted}

	The presence of the proxy is to conveniently access the type later on.
	The methods \texttt{argsTy} and \texttt{outputTy} of the type class
	\texttt{Family f} discussed previously requires to produce \texttt{TList}
	for each argument and output type, therefore automatically requiring
	an instance of \texttt{a :<: f} for any type directly or indirectly mentioned. 
	Moreover	these instances ensure also their presence in \texttt{TypesOf f},
	needed to fulfill the \texttt{getElem} method.
	
	An appropriate \texttt{TList} can be automatically built for lists of types 
	known at compile time, in a similar manner to what happens for
	\texttt{SList} with \texttt{KnownSList}, as described in \ref{?}.
	\todo{Specific section about SList.}
	
\begin{minted}{haskell}
class KnownTList f as where
  tlist :: TList f as
\end{minted}
	The only two generic instances are:
\begin{minted}{haskell}
instance KnownTList f '[] where
  tlist = TNil

instance (a :<: f, KnownTList f as) => KnownTList f (a ': as) where
  tlist = TCons Proxy tlist
\end{minted}

	The technique exploits the automatic instance resolution that happens
	while type-checking, to progressively build the desired \texttt{TList}.
	In the second instance the type checker infer the expected type 
	\texttt{Proxy a} for \texttt{Proxy} and the recursive call to \texttt{tlist} is 
	justified by the constraint  \texttt{KnownTList f as} in the context. 
	The constraint \texttt{a :<: f} is 	instead required by	the constructor 
	\texttt{TCons}.
		
	It is also entirely straightforward to provide an instance for the 
	type	class \texttt{Reify}, which converts a \texttt{TList f as} 
	into the corresponding singleton type 
	\texttt{SList as}, introduced in \ref{slist section}. 
	\todo{Add section specific to SList}

\begin{minted}{haskell}
instance Reify (TList f) where
  toSList TNil = SNil
  toSList (TCons _ t) = SCons (toSList t)
\end{minted}
	
	\paragraph{DList}
	The type \texttt{DList f as} represents a list of \texttt{DTree} of types
	determined by 	\texttt{as}. Their definition is entirely similar to
	the one given in the formal model, except that they are both parametric
	in the family \texttt{f}. Similarly to \texttt{TList}, each type contained
	in the list is forced to belong to \texttt{f}, using the constraint 
	\texttt{a :<: f}.
	
\begin{minted}{haskell}
data DList f xs where
  DNil :: DList f '[]
  DCons :: (a :<: f) => DTree f a -> DList f as -> DList f (a ': as)

data DTree f a where
  Node :: f as a -> DList f as -> DTree f a
\end{minted}
	
	The data type \texttt{DTree f a} is a type safe representation of
	an algebraic data type of type \texttt{a}.
	
	\section{Diff}
	This section describes the implementation of the 
	\texttt{diff} algorithm presented in 	section \ref{subsec:Diff} 
	and the related data types. 
	
	\subsection{Edit Script}
	The edit script data type described in \ref{subsec:EditScript} consists in
	a well-typed list of edits of type \texttt{u \textasciitilde> v}, each of them
	encoding the change made with their source and target values.
	That definition is particularly elegant because it isolates the actual set of
	edits from the conditions required to stack them in a type-safe manner,
	which are embedded in the cons constructor signature and consists
	in expecting the input types of the edit to match the prefix of the
	type lists of the rest of the edit script.
	Remarkably these rules are the same regardless of the concrete edit 
	at hand, which makes it possible to separate the two data types.
	This modular representation is unfortunately not possible in the current
	version of GHC for two reasons. Firstly, as Yorgey et. al explain in 
	\cite{Yorgey12}, GADTs are unpromotable,  i.e. it is not possible index
	a GADT by another GADT, which prevents the edit data type
	to be indexed by values.
	This feature would require kind equality and kind coercions, which would 
	dramatically complicate type equivalence.
	Secondly not even indexing edits directly by their input and output type lists
	would completely solve the issue:
	
\begin{minted}{haskell}
data Edit f as bs cs ds where
  EIns :: f as a -> Edit f [] [] as [ a ] 
  EDel :: f as a -> Edit f as [ a ] [] [] 
  EUpd :: f as a -> f bs a -> Edit f as [ a ] bs [ a ] 

data Edits f xs ys where
  ENil :: Edits f [] [] 
  ECons :: Edit f as bs cs ds -> Edits f (as :++: xs) (cs :++: ys) -> Edits f (bs :++: xs) (ds :++: ys) 
\end{minted}
	
	The definition of \texttt{Edits} is rejected because the type parameters
	\texttt{xs} and \texttt{ys} are ambiguous. The problem is that 
	in general it is not possible to invert type families, therefore
	the type checker refuses to find suitable parameters \texttt{xs} and 
	\texttt{ys}, to solve unification problems such as \texttt{zs = as :++: xs},
	for some given \texttt{zs}. This issue could be resolved including
	two singleton lists \texttt{SList xs} and \texttt{SList ys} to \texttt{ECons},
	which would however slightly obfuscate the code.
	Note that this is not an issue in Agda data types because parameters
	must all be declared either as implicit or explicit arguments and 
	hence are always available, which is the essence of the fix proposed
	using singleton lists.	
	
	To sidestep all these limitations the edit script data type have been defined 
	similarly to that of \cite{Lemp09}, collapsing edit and edit script into
	a single data type, introducing specific constructors for
	each kind of edit. Note that this definition completely preserves type safety,
	but it is only more repetitive.

\todo{Margin fix}
\begin{minted}{haskell}
data ES f xs ys where
  Ins :: (a :<: f) => f xs a -> ES f ys (xs :++: zs) -> ES f ys (a ': zs)
  Del :: (a :<: f) => f xs a -> ES f (xs :++: ys) zs -> ES f (a ': ys) zs
  Upd :: (a :<: f) => f xs a -> f ys a -> ES f (xs :++: zs) (ys :++: ws) -> ES f (a ': zs) (a ': ws)
  End :: ES f [] []
\end{minted}

	The edit script data type is also parametric in the family \texttt{f} 
	and likewise enforces that all the types involved belong to \texttt{f},
	with constraints like \texttt{a :<: f}. 

	\subsection{Memoization}
	The \texttt{diff} algorithm described in \ref{subsec:AlgoDiff} is inefficient 
	because same subcomputations are recomputed multiple times.
	An equivalent, but more efficient, version can be achieved using 
	memoization, i.e. storing the result of subcomputations in a lookup-table,
	encoded by the following data type:
	
\begin{minted}{haskell}	
data EST f xs ys where
  NN :: ES f [] [] -> EST f [] []
  NC :: (b :<: f) => f xs b -> ES f [] (b : ys) 
     -> EST f [] (xs :++: ys) 
     -> EST f [] (b : ys)
  CN :: (a :<: f) => f xs a -> ES f (a : ys) [] 
     -> EST f (xs :++: ys) []
     -> EST f (a : ys) []
  CC :: (a :<: f, b :<: f) => f xs a -> f ys b 
     -> ES f (a : zs) (b ': ws) 
     -> EST f (a : zs) (ys :++: ws)
     -> EST f (xs :++: zs) (b : ws)
     -> EST f (xs :++: zs) (ys :++: ws)
     -> EST f (a : zs) (b : ws)
\end{minted}
	
	A table of type \texttt{EST f xs ys} contains an edit script
	of minimal cost of type \texttt{ES f xs ys}, and the subtables
	corresponding to its tail, obtained by placing either an insert, or
	a delete, or an update edit.
	The table is partitioned in four groups depending on the fact
	that the source and target list is empty or not.
	Specifically \texttt{NN} contains the only edit script in which both
	of them are empty, in \texttt{CN} and \texttt{NC} the target and source
	lists are respectively empty and hence contain only one subtable,
	lastly in \texttt{CC} the lists are both non-empty.
	The edit script contained in a table can be easily retrieved:
	
\begin{minted}{haskell}
getDiff :: EST f xs ys -> ES f xs ys
getDiff (NN e) = e
getDiff (NC _ e _) = e
getDiff (CN _ e _) = e
getDiff (CC _ _ e _ _ _) = e
\end{minted}

	The function \texttt{diffT} builds a memoization table recursively:

\begin{minted}{haskell}
diffT :: (Family f, Metric f) => DList f xs -> DList f ys -> EST f xs ys
diffT DNil DNil = NN End
diffT (DCons (Node a as) xs) DNil = CN a (Del a (getDiff d)) d 
  where d = diffT (dappend as xs) DNil
diffT DNil (DCons (Node b bs) ys) = NC b (Ins b (getDiff i)) i
  where i = diffT DNil (dappend bs ys)
diffT (DCons (Node a as) xs) (DCons (Node b bs) ys) = CC a b (best a b i d u) i d u
  where u = diffT (dappend as xs) (dappend bs ys)
          i = extendI a xs u
          d = extendD b ys u
\end{minted}
 
	The only interesting case is the last one, in which 
	both the input lists are non-empty.
	Firstly note there is only \emph{one} recursive call to \texttt{diffT}.
	The recursion is well-founded because both the input lists are consumed 
	removing the nodes \texttt{a} and \texttt{b}.
	From the table so obtained, the alternative tables in which 
	\texttt{a} is inserted or \texttt{b} is deleted are obtained using the functions
	\texttt{extendI} and \texttt{extendD}.
	Among these three options, the script with minimal cost is chosen
	by the function \texttt{best}.
	
\begin{minted}{haskell}
best :: (Metric f, a :<: f, b :<: f)
     => f as a -> f bs b
     -> EST f (a : xs) (bs :++: ys)
     -> EST f (as :++: xs) (b : ys)
     -> EST f (as :++: xs) (bs :++: ys)
     -> ES f (a : xs) (b : ys)
best f g i d c = 
  case decEq f g of
    Just Refl -> Upd f g (getDiff c) & a & b
    Nothing -> a & b
  where a = Del f (getDiff d)
          b = Ins g (getDiff i)
\end{minted}
	
	Both nodes can be consumed by an update only if they have the 
	same type, which is tested by the \texttt{decEq} function.
 	The other alternatives, which are always possible, consists of deleting
 	the first node or inserting the second. Note that the source and target
 	lists of the three tables are the same of their edit script, hence the scripts
 	produced are all well-typed.

	The binary operator \texttt{\&} selects the edit script with minimal cost:

\begin{minted}{haskell}
(&) :: Metric f => ES f xs ys -> ES f xs ys -> ES f xs ys
x & y = if cost x <= cost y then x else y
\end{minted}

	The function \texttt{cost} computes the score of an edit script,
	according to the cost model described in \ref{subsec:AlgoDiff}.

\begin{minted}{haskell}
cost :: Metric f => ES f xs ys -> Double	
cost End = 0
cost (Ins x xs) = 1 + cost xs
cost (Del x xs) = 1 + cost xs
cost (Upd f g xs) = distance f g + cost xs
\end{minted}

	Of course the \texttt{cost} function inefficiently recomputes the cost of an 
	edit script multiple times.
	It is entirely straightforward to adjust the edit script data type or the 
	memoization table to store also the score of each edit script.
	I have decided to omit this optimization from the presentation in order
	to keep it clear	and focused on the type related issues, which are 
	more interesting and challenging.
	
	Lastly the type class \texttt{Metric f} provides an hook to the final user
	for comparing nodes, assigning a numeric distance between them.
\begin{minted}{haskell}
class Metric f where
  distance :: f xs a -> f ys a -> Double
\end{minted}
	Any \texttt{Metric} instance is expected to satisfy the axioms listed 
	in \ref{par:Metric}.	
	
	The function \texttt{extendI} use an auxiliary intermediate 
	data type \texttt{DES} used to existentially quantify the portion of the table
	that needs to be extended.
	
\begin{minted}{haskell}
data DES f a xs ys where
  DES :: f zs a -> ES f (a : xs) ys -> EST f (zs :++: xs) ys -> DES f a xs ys
\end{minted}

	If the source list is non-empty, then it is possible to produce such a
	data type.

\begin{minted}{haskell}
extractD :: EST f (a : xs) ys -> DES f a xs ys
extractD (CN g e i) = DES g e i
extractD (CC f g e _ i _) = DES f e i
\end{minted}

	Note that the type \texttt{a : xs} in the target list restrict the possible
	values of \texttt{EST} only to those listed.

	Finally \texttt{extendI} is defined as follows:
	
\begin{minted}{haskell}	
extendI :: (Metric f, a :<: f)
        => f xs a -> DList f ys -> EST f zs (xs :++: ys) -> EST f zs (a ': ys)
extendI f _ i@(NN e) = NC f (Ins f e) i
extendI f _ i@(NC _ e _) = NC f (Ins f e) i
extendI f _ i@(CN _ _ _) =
  case extractD i of
    DES g e c -> CC g f (best g f i d c) i d c
      where d = extendI f undefined c
extendI f _ i@(CC _ _ e _ _ _) =
  case extractD i of
    DES g e c -> CC g f (best g f i d c) i d c
      where d = extendD f undefined c
\end{minted}

	First of all the second argument of type \texttt{DList f ys} is never inspected
	and it is introduced only to avoid the ambiguity problem discussed 
	previously.
	Extending the target list does not affect the source list, hence
	the tables \texttt{NC} and \texttt{CC} keep the same constructor.
	On the other hand when the target list is empty, in the \texttt{NN} and
	\texttt{CN} case, the constructor is substituted respectively by
	\texttt{NC} and \texttt{CC}.
	In the last two cases the function \texttt{extractD} is used to extract
	the appropriate node \texttt{g}, edit script \texttt{e} and table
	\texttt{c}, needed to compute the recursive extension \texttt{d}
	and lastly select the table with minimal cost.
		
	The function \texttt{extendD} is analogous and thus omitted.

	\subsection{Algorithm}
	\label{subsec:HaskellAlgo}
	Using all the constructs defined previously, the conventional
	interface of GNU diff can be easily made available to the
	user:

\begin{minted}{haskell}
gdiff :: (Metric f, a :<: f, b :<: f) => a -> b -> ES f '[ a ] '[ b ]
gdiff x y = getDiff (diffT dx dy)
  where dx = DCons (toDTree x) DNil
         dy = DCons (toDTree y) DNil
\end{minted}
	
	The inverse function \texttt{patch}, which computes the target
	object from the edit script, is also part of the interface
	provided to the user.

\begin{minted}{haskell}
patch :: Family f => ES f xs ys -> DList f ys
patch (Ins x e) = insert x (target e)
patch (Del x e) = target e
patch (Upd x y e) = insert y (target e)
patch End = DNil
\end{minted}

	The function \texttt{insert} pops the arguments of the 
	constructor from the given \texttt{DList} and builds
	a well-typed \texttt{DTree} which is then pushed on the list.

\begin{minted}{haskell}
insert :: (a :<: f) => f xs a -> DList f (xs :++: ys) -> DList f (a ': ys)
insert x ds = DCons (Node x ds1) ds2
  where (ds1, ds2) = dsplit (reifyArgs x) ds
\end{minted}

	The function \texttt{dsplit} is entirely similar to the function
	\texttt{dsplit} described in \ref{par:UtilityFun}, the first 
	argument is a singleton type needed due to the strict
	phase separation enforced in Haskell.
	The function \texttt{reifyArgs} retrieves the singleton
	type relative to its arguments.

\begin{minted}{haskell}
reifyArgs :: Family f => f xs a -> SList xs
reifyArgs = toSList . argsTy
\end{minted}

	\subsection{Discussion and Related Work}
	The edit script data type, the memoization technique 
	employed in the \texttt{diff} algorithm presented in this section
	are largely inspired by the work on type-safe diff by 
	Lempsink \cite{Lemp09}.
	As remarked in section \ref{subsec:RelatedWork}, the update
	edit used represents a generalization of the original copy edit, 
	and has been preferred because it improves the alignment of nodes, 
	simplifying the merging phase in the \texttt{diff3} algorithm.
	The 

	\paragraph{Kind system}
	The enrichment of the \emph{kind} system proposed in
	\cite{Yorgey12} and available from GHC 7.6
	makes the data type definitions ``well-kinded''
	and reduce the additional machinery needed to deal with them.
	Specifically using the DataKinds extension lists are 
	automatically promoted and allows to index the \texttt{ES}
	and \texttt{DList} data types with lists of types, whose kind is 
	\texttt{[ * ]}.
	Direct pattern match is possible and does not required 
	additional class constrains, such as \texttt{IsList}.
	Furthermore the kind system has been enhanced so that
	the correct kind is automatically inferred, reducing the number
	of kind annotations needed. Most of the time kinds 
	annotations are superfluous and some have been
	added exclusively for documentation purposes.
	The memoization related functions have been greatly 
	simplified using auxiliary intermediate GADTs to
	existentially quantify variables, as opposed to the continuation
	style employed by  Lempsink, in which functions take
	an higher rank function as an additional parameter.
	The two approaches are roughly equally expressive, however
	direct style code is much more readable and concise.
	As a result the implementation proposed in this thesis 
	requires less boilerplate code and it is closer to the
	Agda version, showing how dependently typed programming
	is becoming more and more natural in Haskell.

	\paragraph{Change Detection}
	The technique employed by Lempsink et al. in the 
	change detection phase for structured data is a variation of
	that proposed by Lozano and Valiente to solve the
	Maximum Common Embedded Subtree problem \cite{Lozano}.
	The input trees are flattened to a list of nodes according to the 
	depth-first perorder traversal and then the longest common
	subsequence of them is computed, similarly to what happens
	in GNU diff. The algorithm proposed in this section follows
	a similar technique, with small differences in the list data type 
	processed in a stack like fashion.

	\paragraph{HList vs DList}	
	In \cite{Lemp09} the input lists is an heterogeneous list, whose 
	raw types are progressively deconstructed exposing the 
	reified constructor and pushing its	children on the stack. In this version 
	instead the input data 	types are directly transformed into 
	\texttt{DTree}, which directly expose their nodes and 
	requires the same for all its children. Due to the lazy semantics of Haskell 
	the two transformations should have approximately the same performance,		
	however there are several advantages in the encoding proposed in this
	thesis.
	
	First of all in \cite{Lemp09} constructors are compared implicitly,
	by trying to deconstruct a value of a raw type against the constructor
	at hand. This is achieved using the partial method \texttt{fields} of type,
	adapted to our definitions, \texttt{f as a -> a -> Maybe (HList as)},
	which is included in the type class \texttt{Family} and it is used to
	implement the auxiliary function \texttt{matchConstructor}, in the
	\texttt{diff} algorithm.
	In this thesis instead such method is requires as a primitive,
	specifically in the operator \texttt{=?=} in the type class \texttt{Family} 
	\ref{subsec:Universe}, and explicitly test for equality between nodes.
	It is much more concise to test equality at once using the latter,
	rather than indirectly trying to match the given constructor,
	with all the possible constructors of a given type.
	In addition that method requires the additional type class \texttt{Type} that 
	provides a method \texttt{constructors}, which 
	returns a list of all the constructors of a given type.
	Secondly comparing the nodes directly does not require to 
	explicitly distinguish between concrete and abstract data types, as 
	it happens in \cite{Lemp09}.
	In summary the encoding provided by \texttt{DList} yields a
	minimal and neat design that requires less boilerplate code,
	which is often source of nasty bugs.
	
	\paragraph{Patch}
	The \texttt{patch} function described in \cite{Lemp09} provides the 
	same interface of GNU patch and applies an edit script 
	to the source object, encoded as an heterogeneous list.
	The semantics of delete and consequently copy relies
	on the partial function \texttt{fields}, which deconstruct the input
	value expecting a certain specific constructor and fails ungracefully,
	if this is not the case.
	As a matter of fact the edit script data type already contains enough 
	information to reconstruct \emph{both} the source and target object.
	Section \ref{subsec:HaskellAlgo} and in \ref{subsec:HaskellAlgo} show
	two total function that retrieve both from the edit script alone.
	The formal model heavily relies on this property and the advantage for a 
	practical implementation	are evident.
	Note that 	the same could be achieved in \cite{Lemp09} using
	exclusively the \texttt{insert} and \texttt{apply} function.

	\paragraph{Polymorphic data types}
	Lempsink et al.\ remark that the encoding used in \cite{Lemp09} and here,
	which is inspired by Multirec \cite{Multirec}, is unable to handle 
	polymorphic data types, such as list, which must be restricted to be 
	monomorphic.
	The reason is that this technique relies on type equality proofs,
	which is straightforward for ground types, but subtle for type parameters.
	In fact the outcome of the equality test depends on the specific
	instantiation of the parameter and thus cannot be defined in advance, once
	and for all as it happens in the current implementation.
	One conservative solution is to ignore the possible instantiations of 
	parameters and treat parameters equal only to themselves.
	Consider for instance the following polyomorphic data type:
	
\begin{minted}{haskell}
data Foo a = C1 Int
             | C2 Bool
             | C3 a
\end{minted}
	At compile time it is known that the first argument of every constructor 
	\texttt{C3} is of type \texttt{a} and it would be possible to 
	exploit this piece of information	to match them accordingly.
	Nevertheless this syntactic approach would fail to detect the matching
	that would arise when \texttt{a} is instantiated with either \texttt{Int} or 
	\texttt{Bool}.
	The \texttt{diff} discussed performs deep matching, which ignores
	the position of values in the tree, therefore this technique would
	profoundly change its semantics. Moreover the situation would get even
	worse with two parameters because \emph{each} parameter could not be
	matched with each ground type and the two parameters could not
	be matched against each other even when they are instantiated with the
	same type: the number of matching ignored would increase 
	\emph{exponentially} in the number of free parameters.
	It is then legitimate to wonder whether the \texttt{diff} algorithm, as it
	is here described, is actually a polymorphic function, or if 	it has any 
	meaningful semantics for polymorphic data.
	

	\paragraph{Type Manipulation}
	In order to use \texttt{diff} two instances
	
	
	\section{Diff$_3$}
	\section{Version Control System}

	% TODO remove when compiling the whole thesis	
	\bibliographystyle{plain}
 	\bibliography{../svc}

\end{document}
