\documentclass[../Thesis.tex]{subfiles}

\begin{document}

\chapter{Format Representation}
\label{chapter:FormatRepresentation}

This chapter addresses the problem of finding a unique representation for data formats, which is an
important component of a structure-aware version control 
system.
Section \ref{sec:FormatIntroduction} introduces the problem by  analyzing the issue in more detail \ref{subsec:FormatMotivation} and briefly covers several auxiliary concepts, such as heterogeneous lists \ref{subsec:TypeList} and partial isomorphism \ref{subsec:PartIso}.
The basic format representation is presented in section \ref{sec:Format}, 
then several extensions are discussed in \ref{sec:Extensions}.
Section \ref{sec:FormatConclusion} concludes, summarizing the main 
contributions of this chapter \ref{subsec:Discussion} 
and reviewing related work \ref{subsec:FormatRelatedWork}.

\section{Introduction}
\label{sec:FormatIntroduction} 
 
\subsection{Motivation}
\label{subsec:FormatMotivation}
A format specifies how some data is stored and encoded in a file.
When the format of a file is known, it is possible to decode it and apply the diff algorithm on the data itself, rather than its representation, thus leading to more accurate edit scripts. These are then are applied by a patch algorithm to the decoded data, producing the new version, which is then serialized in a file according to its format specification.
A structure-aware version control system exploits this to
avoid unnecessary conflicts and thus improving 
the quality of revision control.

Formats are usually described using formal grammars, such as context free grammars.
A parser is a function that recognizes a grammar in the input string and extracts the structured data that it describes.
Printing is the opposite function, which consists of serializing some structured data, according to the format specifications.
In functional programming languages parser combinators are commonly used to implement parsers.
Furthermore there are several tools, such as Yacc \cite{YACC} 
and Happy \cite{HAPPY}, called parser generator,
that automatically generate a parser for a given grammar. 

Parsers and printers of a format should always be one the inverse of the other in order to ensure round-trip behavior, i.e.\ printing some data to a string
and running the parser on it should give back the initial value. Similarly parsing a properly formatted file and printing it back should produce the original file.
These two functions are usually defined independently, which has several drawbacks. Firstly it is a repetitive and error-prone task because the two functions follow the same format, secondly, as the format changes, the user is burdened with the task of keeping the two functions synchronized. Lastly there is no actual guarantee that the two functions are indeed each other's inverse.

	\subsection{Type List}
	\label{subsec:TypeList}
	Haskell type system have been enhanced in GHC with
	several extensions. The DataKind extension enables
	user defined kinds, through datatype promotion, i.e.\
	data types are automatically promoted to kinds and
	their constructors to type constructors.
	Among these list are natively promoted to kinds
	and the same syntax used for value lists is lifted
	to types. It should always be clear from the context
	when a list is a value or a type; ambiguous cases
	will be distinguished prefixing types with a quote.
	The PolyKinds extension enables kind polymorphism.
	Kind variables are always universally quantified implicitly,
	furthermore kinds are inferred automatically,
	therefore kind annotations are usually superfluous, 
	however few will be left to help the reader.
	
	\paragraph{Examples}
	The constructors for type level lists are kind polymorphic,
	just like their value counterpart are polymorphic in the type.
\begin{verbatim}
*> :k '[]
'[] :: [k]
\end{verbatim}

	Note however that they are homogeneous with respect to the kinds of 
	their elements.
\begin{verbatim}
*> :k '[Int, Bool]
'[Int, Bool] :: [*]
*> :k '[Maybe, []]
'[Maybe, []] :: [* -> *]
\end{verbatim}
	In the second example \texttt{[]} is the list type constructor.
	
	\paragraph{Type Family}
	Common functions on lists can be lifted to the type level
	using a \emph{closed type family}.

	For example it is possible to append type level lists just like
	it happens for concrete lists.
	The type family contains a recursive call and it is defined
	by induction on the first list, following closely that
	of \texttt{(++) :: [ a ] -> [ a ] -> [ a ]}.
\begin{minted}{haskell}	
type family (:++:) (xs :: [ k ]) (ys :: [ k ]) :: [ k ] where
  '[] :++: ys = ys
  (x ': xs) :++: ys = x ': (xs :++: ys)
\end{minted}

	Note that just like the standard append is polymorphic in \texttt{a}, so its 
	type level counter part is	kind polymorphic. 
	Likewise the input and output lists share the same polymorphic kind 
	\texttt{k}.
	
\begin{verbatim}
*> :kind! '[Int, Bool] :++: '[Char, Double]
'[Int, Bool] :++: '[Char, Double] :: [*]
= '[Int, Bool, Char, Double]
\end{verbatim}
	
	Analogously a type level map is easily defined:
\begin{minted}{haskell}
type family Map (f :: k1 -> k2) (xs :: [ k1 ]) :: [ k2 ] where
  Map f '[] = '[] 
  Map f (x ': xs) = f x ': Map f xs
\end{minted}
	
	For example mapping the type constructor \texttt{[] :: * -> *} over a list
	of types of kind \texttt{*} yields:
	
\begin{verbatim}
*> :kind! Map [] '[Char, Int , Bool]
Map [] '[Char, Int , Bool] :: [*]
= '[[Char], [Int], [Bool]]
\end{verbatim}

	\paragraph{Heterogeneous List}
	\label{par:HLIst}
	The data type \texttt{HList}, introduced by Kiselyov 
	\cite{Kiselyov04stronglytyped}, is an heterogeneous list, 
	indexed by a type level list. 
	It differs from the conventional homogeneous list data type, because it 
	contain values of possibly different types, which are all 
	stored in its index.

\begin{minted}{haskell}
data HList (xs :: [ * ]) where
  Nil :: HList '[] 
  Cons :: x -> HList xs -> HList (x ': xs) 
\end{minted}

	The following functions will be assumed in the following.
	Their implementation is clear from their
	signatures and therefore omitted. 
	Furthermore they are completely analogous to 
	the correspondent functions for homogeneous lists.
	
\begin{minted}{haskell}
hHead :: HList (x ': xs) -> x
hsingleton :: x -> HList '[ x ]
happend :: HList xs -> HList ys -> HList (xs :++: ys)
\end{minted}

	\paragraph{SList}
	\label{par:SList}
	\emph{Singleton types} are needed to support dependently 
	typed programming in languages with a strict phase separation
	between run-time and compile time \cite{Eisenberg12DTPS}, like Haskell.
	More precisely a singleton type is a type with only one non-$\bot$ value 
	and represent a run-time witness of a type \cite{Eisenberg12DTPS}.
	The type \texttt{SList xs} is the singleton type for the type level list 
	\texttt{xs}.
	
\begin{minted}{haskell}
data SList (xs :: [ * ]) where
  SNil :: SList '[]
  SCons :: SList xs -> SList (x ': xs)
\end{minted}

	Singleton types are essential to implement certain kind of functions,
	especially when type families are involved.
	Consider for instance the function \texttt{split}, which
	splits an heterogeneous list in two parts:
\begin{minted}{haskell}
split :: HList (xs :++: ys) -> (HList xs, HList ys)
\end{minted}

	The indexes of the two output lists, \texttt{xs} and \texttt{ys},
	come from the index of the input list, in which they are appended.
	Crucially it is not possible to pattern match directly on it, because
	of the presence of the type family application \texttt{xs :++: ys} in
	its index:
	should it be 	there a case for \texttt{Nil} or \texttt{Cons} ?
	That depends on the result of \texttt{xs :++: ys}, 
	however it is not possible to pattern match on it 
	directly because it is just a type.
	Nevertheless it is possible to implement this function if \texttt{xs}
	is known to be empty or not.
	A singleton type of type \texttt{SList xs}
	introduces a true value, which can be inspected 
	providing this piece of information about its index.
	
\begin{minted}{haskell}
split :: SList xs -> HList (xs :++: ys) -> (HList xs, HList ys)
split SNil hs = (Nil, hs) 
split (SCons s) (Cons h hs) = (Cons h hs1, hs2)
  where (hs1, hs2) = split s hs
\end{minted}
	By pattern matching on it the first part of the list can be distinguished from 
	the second, so that the two can be separated.
	Specifically when \texttt{xs} is empty the rest of the list belongs
	to the second part, otherwise the head of the list belongs
	to the first part.
	
	Often it will be necessary to retrieve a singleton type of a list-indexed data 
	type. The type class \texttt{Reify} is defined for this purpose:
	
\begin{minted}{haskell}
class Reify (f :: [ * ] -> *) where
  toSList :: f xs -> SList xs
\end{minted}

	For example it is easy to make \texttt{HList} instance of \texttt{Reify}:
\begin{minted}{haskell}
instance Reify HList where
  toSList Nil = SNil
  toSList (Cons _ hs) = SCons (toSList hs)
\end{minted}	
	
\subsection{Partial Isomorphism}	
\label{subsec:PartIso}
Parsers do not only recognize a grammar in an input string, but usually
produce an abstract syntax tree, so that the structured data can be
then processed more conveniently. 
Typically abstract syntax trees are represented directly as a domain
specific user-defined data type. 
On the contrary pretty printing requires to deconstruct the abstract syntax tree, in order to serialize each part appropriately.
In order to unify parser and printer in a single entity, constructors
and deconstructors of a data type must be coupled together.
The partial isomorphism data type is used for this purpose.

\begin{minted}{haskell}
data Iso xs ys = Iso { apply    :: HList xs -> HList ys, 
                       unapply  :: HList ys -> Maybe (HList xs)}
\end{minted}

A value of type \texttt{Iso xs ys} represents a partial isomorphism, consisting in the mapping \texttt{apply} and its partial inverse \texttt{unapply}.
The second function is partial because a data type might have
several constructors and therefore the deconstructor of
a specific isomorphism could fail if the value has been
constructed using another constructor.

Partial isomorphisms form a category:
\begin{minted}{haskell}
identity :: Iso xs xs
identity = Iso id Just

(.) :: Iso ys zs -> Iso xs ys -> Iso xs zs
(.) g f = Iso s t
  where s = apply g . apply f
          t = unapply g >=> unapply f
\end{minted}

\paragraph{Example} 
As an example consider the two partial iomorphisms that correspond
to list constructors and deconstructors. 

\begin{minted}{haskell}
nil :: Iso '[] '[ [a] ]
nil = Iso f g
  where f Nil = Cons [] Nil
        g (Cons [] Nil) = Just Nil
        g (Cons (_:_) Nil) = Nothing

cons :: Iso '[a : [a]] '[ [a] ]
cons = Iso f g
  where f (Cons x (Cons xs Nil)) = Cons (x:xs) Nil
        g (Cons [] Nil) = Nothing
        g (Cons (x:xs) Nil) = Just (Cons x (Cons xs Nil))
\end{minted}

\section{Format}
\label{sec:Format}
This thesis addresses the format problem presented in 
\ref{subsec:FormatMotivation} by implementing an embedded domain specific language (EDSL) that describes two inverse semantics at once.
The specification itself, is agnostic to the semantics, which is chosen afterwards instantiating appropriately a type parameter.

Context-free parsers are usually defined using applicative 
and alternative combinators. Since \texttt{Functor} is a superclass of \texttt{Applicative} and \texttt{Applicative} is a superclass of \texttt{Alternative}, any instance of the class \texttt{Alternative} yields
also the \texttt{Functor} and \texttt{Applicative} instances.

\begin{minted}{haskell}
(<$>) :: Functor f => (a -> b) -> f a -> f b
pure :: Applicative f => a -> f a
(<*>) :: Applicative f => f (a -> b) -> f a -> f b
(<|>) :: Alternative f => f a -> f a -> f a
empty :: Alternative f => f a
\end{minted} 
% $

Each of this combinator will be encoded as a constructor of the \texttt{Format} data type. With an abuse of notation, the operators \texttt{<\$>}, \texttt{<*>}, \texttt{<|>}, will be used as the corresponding \texttt{Format} constructors. 
It should always be clear from the context which is which. 
To keep the description clear, I will firstly discuss formats with token of type \texttt{Char} and stream of type \texttt{String}, then I will show the changes needed to support any kind of token and stream.

\begin{minted}{haskell}
data Format (f :: * -> *) (xs :: [ * ]) where
\end{minted}

A value of type \texttt{Format f xs} has underlying semantics \texttt{f} and involve values of types \texttt{xs}. Furthermore
an instance of \texttt{Reify (Format f)} is assumed.

The two inverse semantics are represented as an interpretation of this data type.
\begin{minted}{haskell}
mkParser :: Alternative f => Format f xs -> f (HList xs)
mkPrinter :: Alternative f => Format f xs -> HList xs -> f String
\end{minted}

\subsection{Functor Format}
The constructor corresponding to the functor combinator \texttt{fmap} is defined as follows:

\begin{minted}{haskell}
data Format (f :: * -> *) (xs :: [ * ]) where
  <$> :: Iso xs ys -> Format f xs -> Format f ys
\end{minted}
%$

The first argument is not a simple function, because we need to store 
a function and its inverse. Therefore the \texttt{Iso} data type is used:


A parser \texttt{f <\$> p} apply the pure function \texttt{f} to the result of the parser \texttt{p}. If the parser \texttt{p} fails, also \texttt{f <\$> p} fails.
This behaviour is accordingly translated in the parsing semantics have exactly:

\begin{minted}{haskell}
mkParser (i <$> f) = apply i <$> mkParser f
\end{minted}

Note that \texttt{apply i} is a pure function.

When printing the format \texttt{i <\$> f}, the input list is deconstructed using the inverse partial isomoprhism (\texttt{unapply}). If it fails the whole printer fails, otherwise the result is used as input to the printer derived from \texttt{f}.

\begin{minted}{haskell}
mkPrinter (i <$> f) hs = 
  case unapply i hs of
    Just ys -> mkPrinter f ys
    Nothing -> empty
\end{minted}
%$

Note that contrary to the standard definition the operator \texttt{<\$>} is defined right associative:

\begin{minted}{haskell}
infixr 4 <$>
\end{minted}
%$

\subsection{Applicative Format}
\label{subsec:ApplicativeFormat}
The constructors corresponding to the applicative combinators are defined as follows:

\begin{minted}{haskell}
data Format f xs where
  ...
  Pure :: HList xs -> Format i xs
  (<*>) :: Format f xs -> Format f ys -> Format f (xs :++: ys)
\end{minted}

The signature of \texttt{<*>} is slightly different from the standard 
one given in the \texttt{Applicative} class shown previously.
This alternative version is actually equivalent, being the  \texttt{$\star$} 
operator from the class \texttt{Monoidal} proposed by McBride and Paterson’s 
\cite{Mcbride08Ape}, revisited to use \texttt{HList} instead of tuples.
Also Fokker employs this version in his lecture notes 
\cite{Fokker}.
This symmetric version is more convenient for our purposes therefore it has been preferred over the original one.

The parsing semantics function is standard:
\begin{minted}{haskell}
mkParser (Pure hs) = pure hs
mkParser (f1 <*> f2) = happend <$> mkParser f1 <*> mkParser f2
\end{minted} 
% $
The format \texttt{Pure hs} is translated to a parser that always returns \texttt{hs} without consuming any input, while the format
\texttt{f1 <*> f2} is converted in a parser that firstly applies the parser of \texttt{f1} and then runs the one of \texttt{f2} on the remaining input. The function \texttt{happend} appends the two lists produced by the parsers toghether, as it is required by the signature of \texttt{<*>}.
The parser \texttt{pure x} never fails, while \texttt{f1 <*> f2}
fails if any of the two fails.

Analogously the semantics for the corresponding printer is given by:
\begin{minted}{haskell}
mkPrinter (Pure _) _ = pure ""
mkPrinter (f1 <*> f2) hs = (++) <$> mkPrinter f1 hs1 <*> mkPrinter f2 hs2
  where (hs1, hs2) = split (toSList f1) hs
\end{minted} 
%$
In the \texttt{pure} case the empty string is returned, because no input is consumed by the corresponding parser.
Similarly in the \texttt{<*>} case the two strings produced by the two formats are appended in the same order, to invert the effect of the corresponding parser.

It is straightforward to show that \texttt{mkPrinter} and \texttt{mkParser} represent each other's inverse for the applicative combinators.

\subsection{Alternative Format}
The constructors that corresponds to the alternative combinators 
have the following signatures:

\begin{minted}{haskell}
data Format f xs where
  ...
  Empty :: SList xs -> Format i xs
  (<|>) :: Format f xs -> Format f xs -> Format f xs
\end{minted}

The parser combinator \texttt{<|>} represents choice between parsers.
The parser \texttt{empty} always fails without consuming any input and it is 
the identity of \texttt{<|>}.

The specific semantics of the choice combinator is most of the time library-dependent and some care is needed to ensure proper invertibility. 
This issue is discusses in more depth in \ref{subsec:Discussion}.

The parsing and printing functions just reuse the alternative combinators
of the underlying semantics.

\begin{minted}{haskell}
mkParser (Empty _) = empty
mkParser (f1 <|> f2) = mkParser f1 <|> mkParser f2

mkPrinter (Empy _) _ = empty
mkPrinter (f1 <|> f2) hs = mkPrinter f1 hs <|> mkPrinter f2 hs
\end{minted}

\subsection{Token Format}
Parsing libraries usually provide a primitive function to retrieve the next token in the stream, whose type signature is usually:
\begin{minted}{haskell}
pSatisfy :: (Char -> Bool) -> Parser Char
\end{minted}

If the next token in the stream satisfies the predicate then it is returned, otherwise the parser fails.
A corresponding constructor is added to \texttt{Format}:

\begin{minted}{haskell}
data Format f xs where
  ...
  Satisfy :: (Char -> Bool) -> Format f '[ Char ]
\end{minted}

The specific primitive used to produce the next token, depends on the on the actual parsing library at hand, therefore a hook is provided in the form of a type class.

\begin{minted}{haskell}
class ParseSatisfy f where
  satisfy :: (Char -> Bool) -> f Char
\end{minted}

The parser semantics is given by:
\begin{minted}{haskell}
mkParser :: (ParseSatisfy f, Alterntaive f) => Format f xs -> f (HList xs)
mkParser (Satisfy p) = hsingleton <$> satisfy p
\end{minted} 
% $

Similarly when printing, a token is produced only if it satisfy the predicate.
\begin{minted}{haskell}
mkPrinter (Satify p) (Cons c Nil) 
  | p c       = pure [ c ]
  | otherwise = empty
\end{minted}

\subsection{Example}
This section presents few examples that can be already implemented with the \texttt{Format} just defined.

\paragraph{Trivial Format}
A \emph{trivial format }is a format parametrized by the empty type-level list.
The parser produced by a trivial format upon success produces the empty \texttt{HList}. The constant partial isomorphism \texttt{ignore} is used to construct trivial formats.

\begin{minted}{haskell}
ignore :: HList xs -> Iso xs '[]
ignore hs = Iso f g (toSList hs) SNil
  where f _ = Nil
         g _ = Just hs
\end{minted}
The isomorphism maps any input to the empty list and maps the empty list back to the given list. As a consequence the printer of a trivial format never fails\footnote{A case which requires special care is discussed in section \ref{subsec:Many}}.

Essentially a trivial format contains only static information: its
parser merely checks the presence of a certain pattern, and its printer
just outputs it.

An example of trivial format is the format \texttt{char c} which recognizes a specific character.

\begin{minted}{haskell}
char :: Char -> Format f '[]
char c = ignore (hsingleton c) <$> satisfy (c ==)
\end{minted}
%$

The trivial format \texttt{string s} recognizes a specific string and can be obtained by repeatedly applying \texttt{char}.

\begin{minted}{haskell}
string :: String -> Format f '[]
string [] = unit
string (c:cs) = char c <*> string cs
\end{minted}

The trivial format \texttt{unit} always succeeds producing
the empty \texttt{HList}. It does not consume any input when parsing and that does not output nothing at all when printing.
\begin{minted}{haskell}
unit :: Format f '[]
unit = Pure Nil
\end{minted}

Other two useful combinators, included in the \texttt{Applicative} class are:
\begin{minted}{haskell}
(<*) :: f a -> f b -> f a
(*>) :: f a -> f b -> f b
\end{minted}
They are used to run a parser and then discard its result.
In this setting, these combinators would have the following signatures:
\begin{minted}{haskell}
(<*) :: Format f xs -> Format f ys -> Format f xs
(*>) :: Format f xs -> Format f ys -> Format f ys
\end{minted}
This degree of generality cannot be achieved because, due to the double  semantics embedded in \texttt{Format}, the discarded values must be printed, in order to correctly invert the parser.
However it is possible to discard trivial formats, because they do not require any input value.
The operator \texttt{(<*)} is implemented as:
\begin{minted}{haskell}
(<*) :: Format f xs -> Format f '[] -> Format f xs
p <* q = 
  case rightIdentityAppend (toSList p) of
    Refl -> p <*> q
\end{minted}
First of all note that this operator is simply implemented as \texttt{<*>}, thus
this operator works only at the type level.
However the expression \texttt{p <*> q} alone is rejected by the 
type checker: counterintuitively \texttt{xs} does not unify with \texttt{xs :++: []}.
In order to convince the type checker of this fact, a \emph{proof} about \texttt{:++:} is required.
The function \texttt{rightIdentityAppend} produces for any type-level list \texttt{xs} the proof that the empty list \texttt{'[]} is right identity to append.
The proof is by induction on \texttt{xs} and hence require its singleton type:
\begin{minted}{haskell}
rightIdentityAppend :: SList xs -> xs :++: '[] :~: xs
rightIdentityAppend SNil = Refl
rightIdentityAppend (SCons s) = 
  case rightIdentityAppend s of
    Refl -> Refl
\end{minted}

The data type \texttt{a :\textasciitilde: b} represents propositional equality. Pattern matching on it brings into scope the constraint \texttt{a \textasciitilde\ b}: a proof for the type checker that \texttt{a} is equal to \texttt{b}.

\begin{minted}{haskell}
data a :~: b where
  Refl :: a :~: a
\end{minted}

A similar reasoning applies for \texttt{*>}.

\subsection{Kleene operators}
\label{subsec:Many}
In parsing libraries the combinator \texttt{many} corresponds to the Kleene star operator and the combinator \texttt{some} to Kleene plus. The parser \texttt{many p} succeeds if \texttt{p} succeeds zero or more times. Similarly the parser \texttt{some p} succeeds if \texttt{p} succeeds at least once.
The default implementation of these combinators is part of the \texttt{Alternative} class. 
\begin{minted}{haskell}
some :: Alternative f => f a -> f [ a ]
some p = (:) <$> p <*> many p

many :: Alternative f => f a -> f [ a ]
many p = some p <|> pure []
\end{minted}
%$

Their format counterpart has the following signature:
\begin{minted}{haskell}
many, some :: Format f xs -> Format f (Map [] xs)
\end{minted}

Their definition follows the same mutually recursive pattern:
\begin{minted}{haskell}
some f = allCons (toSList f) <$> f <*> many f
many f = some f <|> allEmpty (toSList f) <*> unit
\end{minted}
% $

The partial isomorphisms \texttt{allEmpty} and \texttt{allCons} have the following signatures:

\begin{minted}{haskell}
allEmpty :: SList xs -> Iso [] (Map [] xs)
allCons :: SList xs -> Iso (xs :++: (Map [] xs)) (Map [] xs)
\end{minted}

% allEmpty SNil = identity SNil
% allEmpty (SCons s) = nil *** allEmpty s
% \end{minted}

% The partial isomorphism \texttt{***} joins two isomorphisms appending
% in order their output/input lists.
% \begin{minted}{haskell}
% (***) :: Iso xs ys -> Iso zs ws -> Iso (Append xs zs) (Append ys ws)
% \end{minted}

The isomorphism \texttt{allCons} corresponds to \texttt{(:)} for simple lists.
When applied it merges the heads (\texttt{xs}) with the corresponding tails (\texttt{Map [] xs}) applying repeatedly the partial isomorphism \texttt{cons}.
When unapplied it splits the given lists (\texttt{Map [] xs}), in heads (\texttt{xs}) and tails (\texttt{Map [] xs}) using the \texttt{cons} deconstructor (\texttt{unapply}). If any of the lists is empty the isomorphism fails.
Lastly it appends the two resulting \texttt{HList}.

Similarly the isomorphism \texttt{allEmpty}, when applied, produces an \texttt{HList} of empty lists (\texttt{Map [] xs}), applying repeatedly the partial isomorphism \texttt{nil}. When unapplied produces the empty \texttt{HList} after checking with the deconstructor \texttt{nil} that all the lists are actually empty. If any of them is non-empty the isomorphism fails.

Note that if the parser produced by \texttt{some} and \texttt{many} succeeds, it will produce lists of the same length. Similarly the corresponding printer will fail if the lists provided have different lengths.

\paragraph{Termination}
\label{par:Termination}
Particular care must be taken when combining \texttt{many} and \texttt{some}
with \emph{trivial} formats. Applying either \texttt{many} or \texttt{some} to a trivial format results in a trivial format, because \texttt{Map [] '[]} equals \texttt{'[]}, like for instance:
\begin{minted}{haskell}
spaces :: Format f '[]
spaces = many (char ' ')
\end{minted}

The parser for this format shows the desired behavior, however the corresponding printer loops, without producing any spaces.

In fact the mutually recursive definition of \texttt{many} and \texttt{some},  terminates only under
the condition that the repeated action will eventually fail.
For parsers this condition can be easily checked.
If the parser \texttt{p} consumes at least one token, the parser \texttt{many p} will eventually terminate: since the input string is finite, \texttt{some p} 
can succeed only a finite number of times, after which only the alternative \texttt{pure []} will succeed.
On the contrary if \texttt{p} does not consume any input the parser
\texttt{many p} will hang.

Note that this behaviour is exactly the same for any type
instance of \texttt{Alternative} that use the default implementation
of \texttt{many} and \texttt{some}:
\begin{verbatim}
*> many (pure ⊥) :: Maybe [a]
*** Exception: <<loop>>
\end{verbatim}

Analogously, when printing the format \texttt{many p} of type \texttt{Format f (Map [] xs)}, the lists provided are always finite, therefore the \texttt{p} printer can succeed only a finite number of time and the function will eventually terminate.
However when \texttt{xs} is empty, i.e.\ \texttt{'[]}, the \emph{trivial} format \texttt{p} always succeeds, which, combined with \texttt{many} and \texttt{some} leads to non-termination.
As a result the \texttt{spaces} printer hangs, while building an infinite list of spaces.

In order to ensure termination also for these formats, the inverse 
semantics for \texttt{many} and \texttt{some} when combined with \emph{trivial formats} has been adjusted as follows. 

The function \texttt{atMost n f k} is used to override the behavior of
possibly non-terminating combinators such as \texttt{many} and \texttt{some}, represented by \texttt{k}, precisely allowing to \texttt{unapply} the given format \texttt{f} at most \texttt{n} times.

\begin{minted}{haskell}
atMost :: Int -> Format f [] 
       -> (forall xs . Format f xs -> Format f (Map [] xs))
       -> Format f []
atMost n f k = ignore hs <$> (k  (f *> Pure hs))
  where  hs :: HList '[ [a] ]    
          hs = hsingleton (replicate n ⊥)
\end{minted}
%$

Firstly the \emph{trivial} format \texttt{f} is transformed in a \emph{non-trivial} format with \texttt{f *> Pure hs}. As a result also the format \texttt{k ((f *> Pure hs)} is \emph{non-trivial} and thus terminating.
Note that the presence of \texttt{Pure} does not affect parsing nor printing, which is instead carried out by \texttt{f}. Using \texttt{ignore hs} the \emph{non-trivial} format is lastly transformed back in a \emph{trivial} format.
The list \texttt{hs} contains \texttt{n} bottom objects of type \texttt{a} and it
is responsible for the peculiar semantics of \texttt{atMost}.
The number of objects contained in this list corresponds to the number of times that the format \texttt{f} is unapplied when printing. Since \texttt{f} is \emph{trivial} it will not inspect these values, therefore, thanks
to Haskell lazy semantics, no run-time failure will occur.

The combinators \texttt{many} and \texttt{some} are redefined as follow.
\begin{minted}{haskell}
many :: Format f xs -> Format f (Map [] xs)
many f = 
  case toSList f of
    SNil -> atMost 0 f manyPrim
    _    -> manyPrim f

some :: Format f xs -> Format f (Map [] xs)
some f = case toSList f of
    SNil -> atMost 1 f somePrim
    _    -> somePrim f
\end{minted}

The functions \texttt{manyPrim} and \texttt{somePrim} correspond to the previous definition of \texttt{many} and \texttt{some}.
The semantics of \texttt{many} and \texttt{some} for non-trivial formats is unchanged, while \emph{trivial} formats will be printed \texttt{f} respectively 0 and 1 times.
The motivation for this semantics is to output the shortest string that can be matched by the corresponding parser.
Furthermore note that the function \texttt{atMost} can also be used to fine-tune arbitrary formats, for example to implement pretty-printers.

\paragraph{Example}
Consider these two \emph{trivial} formats:
\begin{minted}{haskell}
spaces0 :: Format f []
spaces0 = many (char ' ')

spaces1 :: Format f []
spaces1 = some (char ' ')
\end{minted}

When printed they will produce a finite string, respectively containing zero and
one space, instead of hanging:
\begin{minted}{haskell}
*> mkPrinter spaces0 Nil :: Maybe String
Just ""
*> mkPrinter spaces1 Nil :: Maybe String
Just " "
\end{minted}

\section{Extensions}
\label{sec:Extensions}
In this section several extensions are incrementally added to the core framework discussed previously. They make the library 
more flexible and usable in practice.

\subsection{Monadic Format}
\label{subsec:Monadic}
Formats often need to be self-contained, therefore they include meta-data in the form of tags, magic numbers and headers, needed to decode the rest of a file correctly. 

\paragraph{Example}
The NetPbm is a family of formats that encodes images as a bitmap. 
The Portable BitMap (PBM) includes a header that contains a magic number (P1) and two numbers $n$ and $m$ that represents the dimensions of the image. The header is followed by a bitmap of of $n \times m$ bits, which encodes the color of each bit (0 for white, 1 for black). In order to parse each row in the bitmap correctly, its dimensions must be taken into account.

A context-sensitive grammar cannot be recognized by applicative (context-free) parsers, but requires monadic parsers, which are 
strictly more expressive. The need of such class of parsers motivates this extension.

Context-sensitive parsers are instance of the \texttt{Monad} class:

\begin{minted}{haskell}
class Applicative m => Monad m where
  (>>=) :: m a -> (a -> m b) -> m b
  return :: a -> m a
  fail :: String -> m a
\end{minted}

The function \texttt{return} lifts a value into the monad and \texttt{fail}
aborts the computation with an error message.
The binding operator \texttt{>>=} allows to inspect the value of the first computation and take actions based on it.

The data type \texttt{Format} is extended with these new constructors:

\begin{minted}{haskell}
data Format f xs where
  ...
  Return :: HList xs -> Format f xs
  Fail :: String -> Format f xs
  (>>=) :: Format f xs -> (HList xs -> Format f ys) -> Format f (xs :++: ys)
\end{minted}

The only remarkable difference is the type signature of the \texttt{Bind} constructor, which encodes \texttt{(>>=)}. 
The resulting type-level list is \texttt{xs :++: ys},
instead of only \texttt{ys}, because the input values relative to the first parser 
must be printed back, in order to correctly invert this parser.

%In addition the \texttt{Bind} constructor takes as an additional argument,
%the singleton type relative to the second format (\texttt{SList ys}).
%This  is needed in order to \texttt{split} 
%the input list  when printing. Even though \texttt{Format} is part of the class
%\texttt{Reify}, an explicit object is needed because the bind operator is high-order, therefore the second format, indexed by \texttt{ys}, is out of reach.
%
%In order to implement monadic formats naturally a smart constructor for \texttt{Bind} is defined:
%\begin{minted}{haskell}
%(>>=) :: KnownSList ys => Format m xs -> (HList xs -> Format m ys) 
%       -> Format m (Append xs ys)
%m >>= k = Bind sList m k
%\end{minted}

The parser and printing semantics are adapted to include also the \texttt{Monad} constraint and the new cases are defined accordingly.
\begin{minted}{haskell}
mkParser :: (Monad m, Alternative m) => Format m xs -> m (HList xs)
mkParser (Return hs) = return hs
mkParser (Fail msg) = fail msg
mkParser (f1 >>= k) = do
  hs1 <- f1
  hs2 <- k hs1
  return (happend hs1 hs2)

mkPrinter :: (Monad m, Alternative m) => Format m xs -> HList xs -> m String
mkPrinter (Return _) hs = return ""
mkPrinter (Fail msg) _ = fail msg
mkPrinter (f1 >>= k) hs = (++) <$> mkPrinter f1 hs1 <*> mkPrinter f2 hs2
  where (hs1, hs2) = split (toSList f1) hs
          f2 = k hs1 
\end{minted}
% $

\paragraph{Example}
The implementation of the PBM format is given in this paragraph
as an example of monadic formats.
For ease of exposition the text-based encoding is considered.

\begin{minted}{haskell}
pbmFormat :: Format m '[Int, Int, [[Char]]]
pbmFormat = pbmHeader >>= \(Cons n (Cons m Nil)) -> pBitmap n m
\end{minted}

The header of a pbm file contains the magic number and two integers that represent the number of rows and columns of the bitmap.
	
\begin{minted}{haskell}
pbmHeader :: Format m '[Int, Int]
pbmHeader = p1 *> int <*> (whitespace *> int <* whitespace)
  where p1 = string "P1" *> whitespace 
\end{minted}

Since the magic number is statically know, \texttt{p1} is a \emph{trivial} 
format.
The format \texttt{int} is a simple formats that converts a non-empty string of digits into an integer.
The format \texttt{whitespace} consumes white space characters when 
parsing and outputs a single space when printing:
\begin{minted}{haskell}
whitespace :: Format m []
whitespace = some (char ' ' <|> char '\n' <|> char '\r' <|> char '\t')
\end{minted}

The bitmap is parsed row by row, exploiting the information provided by the header.
\begin{minted}{haskell}
pBitMap :: Int -> Int -> Format m '[ [[Char]] ]
pBitMap n m = count n (count m (bit <* whitespace))
  where bit = oneOf "01"
\end{minted}

The format \texttt{count n f} replicates the format \texttt{f} exactly \texttt{n} times.
\begin{minted}{haskell}
count :: Int -> Format f xs -> Format f (Map [] xs)
count n f
  | n <= 0    = allEmpty (toSList f) <$> unit
  | otherwise = allCons (toSList f) <$> f <*> count (n - 1) f
\end{minted}

Note that, unlikely \texttt{many} and \texttt{some}, the format combinator 
\texttt{count} is well-behaved also when applied to \emph{trivial} formats, 
because it does \texttt{unapply} the format \texttt{f} a finite number of times.

The format \texttt{oneOf} matches any of the characters listed and returns it.

\begin{minted}{haskell}
oneOf :: [Char] -> Format f Char
oneOf cs = Satisfy (`elem` cs) 
\end{minted}

\subsection{Token and Stream}
\label{subsec:TokenAndStream}
Formats are roughly divided in text and binary formats, which determine what type of tokens needs to be recognized and consequently what kind of stream 
must be provided. As a consequence the framework presented must be adapted to be parametric in the token and stream type.

The \texttt{Format} data type is firstly extended with an additional type parameter that encodes the token type. The constructor \texttt{Satisfiy} is then
adjusted accordingly.

\begin{minted}{haskell}
data Format (f :: * -> *) (i :: *) (xs :: [ * ]) where
  (<$>) :: Iso xs ys -> Format f i xs -> Format f i ys
  Pure :: HList xs -> Format f i xs
  (<*>) :: Format f i xs -> Format f i ys -> Format f i (xs :++: ys)
  Empty :: SList xs -> Format f i xs
  (<|>) :: Format f i xs -> Format f i xs -> Format f i xs
  Satisfy :: (i -> Bool) -> Format f i [ i ]
  Return :: HList xs -> Format f i xs
  Fail :: String -> Format f i xs
  (>>=) :: Format f i xs -> (HList xs -> Format f i ys) -> Format f i (xs :++: ys)
\end{minted}
%$

Similarly another parameter is added to the \texttt{ParseSatisfy} type class
\begin{minted}{haskell}
class ParseSatisfy f i where
  satisfy :: (i -> Bool) -> f i
\end{minted}

In the printing semantics the tokens produced must be combined to produce a stream. The type class \texttt{PrintToken f i s} provides the
method \texttt{printToken}, which transforms a single token
of type \texttt{i} in a printer of the stream type \texttt{f s}.
\begin{minted}{haskell}
class PrintToken f i s where
  printToken :: i -> f s 
\end{minted}

Furthermore it is reasonable to expect that the stream type is 
an instance of \texttt{Monoid}, as it happens for common stream
types such as \texttt{String}, \texttt{ByteString} and lists.
The type class \texttt{Monoid} provides the methods 
\texttt{mconcat} and \texttt{mempty} which can be conveniently employed
in the semantics of the applicative combinators and monadic combinators.

\begin{minted}{haskell}
mkPrinter :: (PrintToken f i s, Monoid s, Alternative f, Monad f) 
           => Format f i xs -> f s
mkPrinter (Satisfy p) (Cons x Nil)
  | p x       = printToken x
  | otherwise = empty
mkPrinter (Pure _) _  = pure mempty
mkPrinter (f1 <*> f2) hs = mappend <$> mkPrinter f1 hs1 <*> mkPrinter f2 hs2
  where (hs1, hs2) = split (toSList f1) hs
mkPrinter (Return _) _  = return mempty
mkPrinter mkPrinter (f1 >>= k) hs = mappend <$> mkPrinter f1 hs1 <*> mkPrinter f2 hs2
  where (hs1, hs2) = split (toSList f1) hs
          f2 = k hs1			
\end{minted}
%$

\subsection{Extensible Format}
\label{subsec:ExtensibleFormat}
The monadic extension discussed in \ref{subsec:Monadic} shows a shortcoming of the current \texttt{Format} representation.
In order to provide new primitives the \texttt{Format} data type 
has been changed, adding new constructors, the semantics functions \texttt{mkParser} and \texttt{mkPrinter} have been adjusted to include the new cases and lastly
the set of constraints required to implement them increased.
The last change is particularly troublesome, because it brakes the support for non-monadic parsers and printers.

The problem lies in the fact that the data type \texttt{Format} is
a \emph{closed universe}, therefore it cannot be arbitrarily extended,
without updating existing code.
This design problem is known as the expression problem \cite{Wadler98ExprPr} and tests the expressivity of programming language.
It requires to define a data type which can be extended adding new cases and adding new functions over it, without recompiling existing code and while retaining static type safety.
A number of solutions have been proposed in literature, but an alternative approach to the problem is presented in this section.
For simplicity the solution is explained using the original simple
example, while the necessary changes to the \texttt{Format} data type
are shown in \ref{subsec:FormatRevised}.

\paragraph{Expression Problem}
The data type \texttt{Expr} represents an arithmetic expression:
\begin{minted}{haskell}
data Expr = Val Int | Add Expr Expr
\end{minted}

The function \texttt{eval} computes the value of an arithmetic expression,
\begin{minted}{haskell}
eval :: Expr -> Int
eval (Val i) = i
eval (Add e1 e2) = eval e1 + eval e2
\end{minted}

Other semantic functions can be defined without modifying existing code.
For instance the function \texttt{pretty} returns a string representation of an expression.

\begin{minted}{haskell}
pretty :: Expr -> String
pretty (Val i) = show i
pretty (Add e1 e2) = "(" ++ show e1 ++ " + " ++ show e2 ++ ")"
\end{minted}

However adding a new constructor to \texttt{Expr} requires to update
all the functions defined over \texttt{Expr} to handle the new case.

To solve the expression problem a separate data type is defined for 
each constructor of the original universe.
An additional type parameter is added to each of them.

\begin{minted}{haskell}
data Val c where
  Val :: Int -> Val c
  
data Add c where
  Add :: (c a, c b) => a -> b -> Add c
\end{minted}

The index \texttt{c} has kind \texttt{* -> Constraint}.
The constraint kind is one of the latest extension to the Glasgow Haskell Compiler (GHC), which provides the distinct kind \texttt{Constraint}.
Equality constraints and type classes have kind \texttt{Constraint}.

Instead of building expressions using the constructors directly, the following smart constructors are used:

\begin{minted}{haskell}
val :: Int -> Val c
val = Val

add :: (Use a c, Use b c) => a c -> b c -> Add c
add = Add 
\end{minted}

The smart constructor \texttt{add} might look redundant at first sight, 
however its signature is very important, because it restricts the arguments of 
\texttt{Add} to be indexed by the same parameter \texttt{c} of kind \texttt{* -> Constraint}.
The type synonym \texttt{Use} ensures that the two arguments
satisfy the constraints required by \texttt{Add}:
\begin{minted}{haskell}
type Use a c = c (a c)
\end{minted}

In general the constraint \texttt{Use a c} requires the presence of an instance for \texttt{c (a c)} for any constraint-indexed type \texttt{a}.

Arbitrary arithmetic expressions can be composed using the smart constructors, while leaving the constraint parameter abstract:
\begin{minted}{haskell}
foo :: (Use Val c, Use Add c) => Add c
foo = val 0 `add` val 1 `add` val 2
\end{minted}

The type signature is mandatory: since \texttt{c} is not instantiated, the 
constraints introduced by the smart constructors must be propagated. As a result each definition will carry a number of class constraints that expose the pieces of expression used.
Note however that the user does not have to manually keep track of them, because they are automatically inferred by the type checker, which
will trigger a type error if any is missing. Furthermore letting the type-checker infer them is very convenient, for example no duplicated constraints are generated and the order of the constraints themselves is insignificant.
For instance in the example \texttt{foo}, the smart constructors \texttt{val} and \texttt{add} are used more than once, however only one constraint per type is inferred. 

Semantics functions are defined by means of a properly kinded type class:

\begin{minted}{haskell}
class Eval a where
  eval :: a -> Int
\end{minted}

The type class \texttt{Eval} has one parameter of kind \texttt{*}, therefore
its kind is \texttt{* -> Constraint}, hence
it can be used to instantiate the constraint parameter \texttt{c}.

New instances are defined separately for each data type. For the base cases the parameter \texttt{c} is actually just a phantom type.

\begin{minted}{haskell}
instance Eval (Val c) where
  eval (Val i) = i
\end{minted}

Instead for the recursive cases the parameter \texttt{c} must be instantiated with the same type class that is being implemented. 

\begin{minted}{haskell}  
instance Eval (Add Eval) where
  eval (Add e1 e2) = eval e1 + eval e2
  \end{minted}

This produces the constraint \texttt{c \textasciitilde\ Eval}, 
which after pattern matching on \texttt{Add}, brings into scope 
the instances \texttt{Eval a} and \texttt{Eval b}, enabling the 
recursive calls of \texttt{eval} on the two subexpressions.
Lastly instead of using \texttt{eval} directly an helper function is needed.

\begin{minted}{haskell}
evalExpr :: Use a Eval => a Eval -> Int
evalExpr = eval
\end{minted}

This function has the only purpose to fix the constraint parameter \texttt{c} 
of indexed expression to \texttt{Eval}.
In fact the smart constructors always leave \texttt{c} abstract, because the same expression may be interpreted by different semantic functions.
Since the parameter of \texttt{Eval} is of kind \texttt{*}, if \texttt{eval}
was used directly on a constraint-indexed expressions, its parameter
would be ambiguous, resulting in a type error.

\paragraph{Adding new functions}
Adding new semantics functions is as simple as declaring a new type class and 
implementing the corresponding instance for each data type.
\begin{minted}{haskell}
class Pretty a where
  pretty :: a -> String

instance Pretty (Val c) where
  pretty (Val i) = show i  

instance Pretty (Add Pretty) where
  pretty (Add e1 e2) = "(" ++ pretty e1 ++ " + " pretty e2 ++ ")"

prettyExpr :: Use a Pretty => a Pretty -> String
prettyExpr = pretty
\end{minted}

\paragraph{Adding new cases}
Adding a new case requires to define a new data type and relative
smart constructor:
\begin{minted}{haskell}
data Mul c where
  Mul :: (c a, c b) => a -> b -> Mul c

mul :: (Use a c, Use b c) => a c -> b c -> Mul c
mul = Mul

bar :: (Use Val c, Use Add c, Use Mul c) => Mul c
bar = foo `mul` foo
\end{minted}

As required by the expression problem, this solution retains static type safety:
when an expression is evaluated by some semantic function,
if for some piece of expression the corresponding instance has
not been provided, a compile-time error will be triggered.

\begin{verbatim}
*> evalExpr bar
No instance for (Eval (Mul Eval)) arising from a use of `evalExpr’
  In the expression: evalExpr bar
\end{verbatim}

Furthermore no recompilation is needed, because each instance is independent from the others.

\paragraph{Discussion}
Several solutions to the expression problem have been proposed.
A naive solution to the problem consists in defining each case as a separate data type, exploiting parametric polymorphism for the recursive ones.

\begin{minted}{haskell}
data Val = Val Int
data Add a b = Add a b
\end{minted}

Likewise semantic functions are defined using type classes.

\begin{minted}{haskell}
class Eval a where
  eval :: a -> Int

instance Eval Val where
  eval (Val i) = i

instance (Eval a, Eval b) => Eval (Add a b) where
  eval (Add e1 e2) = eval e1 + eval e2
\end{minted}

The instances for the recursive cases are defined assuming an appropriate 
context, that brings in scope the instances for the children, therefore allowing
the recursive call to \texttt{eval}.

The main drawback of this approach is that the structure of each expression is replicated in its type. For instance:

\begin{minted}{haskell}
foo :: Add Val (Add Val Val)
foo = Add (Val 1) (Add (Val 2) (Val 3))
\end{minted}

This encoding is inconvenient not only because of the unwieldy types
that it requires, but also because it restricts the class of well-typed programs that can be defined with it. It is easy to incur in infinite types, even in
non recursive programs.
For instance the following exponentiation function cannot be typed, because it contains an infinite type.

\begin{minted}{haskell}
exp e n = foldr (\_ -> Mul e) e [1..n-1]
\end{minted}

Swiestra represents an extensible data type as a fixed-point of a functor \cite{Swierstra08DTL}. Each constructor is defined as a separate data type
that is injected in a functor and semantic functions consist of algebras,
defined in a piecewise manner using the type class system, which are ultimately \emph{folded} over the functor.
Swiestra develops automatic injectors, which work as smart constructor
and appropriately arrange different data types in a functor.
However this feature, essential to make this technique practical, is fragile because it relies on the controversial overlapping instances extension and requires explicit type signatures in order to deduce the right injection.

Carette et al.  present an alternative approach to the expression problem that exploits only the type-class system \cite{Carette09FTP}. A series of lecture notes \cite{Oleg12FTP} contain an introduction closer to this presentation.

Constructors are transformed in methods of a type class, parametrized by a type, that wraps the result of a semantic function.

\begin{minted}{haskell}
class Expr a where
  val :: Int -> a
  add :: a -> a -> a
\end{minted}

Semantic functions are defined as a data type that wraps the result of
the interpretation and implementing the corresponding 
instance for \texttt{Expr}.

\begin{minted}{haskell}
newtype Eval = Eval {eval :: Int}

instance Expr Eval where
  val n = Eval n
  add e1 e2 = Eval (eval e1 + eval e2)
\end{minted}

New cases are added with a new class.

\begin{minted}{haskell}
class MulExpr a where
  mul :: a -> a -> a

instance MulExpr Eval where
  mul e1 e2 = Eval (eval e1 * eval e2)
\end{minted}

Compared to the approach presented here, this method has few advantages.
Firstly it does not require any particular extension, but exploits uniquely 
the type-class system. Secondly the number of constraints generated can be
reduced via subclassing. For instance it would be natural to make the class
\texttt{MulExpr} subclass of \texttt{Expr}.
One possible disadvantage of this solution is that, once an expression is composed, it cannot be inspected, since methods do not create an actual data type. Naive pattern-match would not work in the solution proposed in this
thesis either, however I conjecture that some useful information could be stored with the constraint parameter. 
Nevertheless both the two solutions fully and effectively solve the original formulation of the expression problem.

\subsection{Format Revised}
\label{subsec:FormatRevised}
The solution proposed in \ref{subsec:ExtensibleFormat} has been
effectively employed to leave the format representation open to
extensions.

\paragraph{Functor Format}
The format data type presented in \ref{subsec:TokenAndStream} have been 
split in several data types and an additional constraint parameter
have been added to each of them.
As a representative example of the this transformation, 
the revised functor format is shown:

\begin{minted}{haskell}
data FMap c (m :: * -> *) (i :: *) (xs :: [ * ]) where
  FMap :: (c m i a) => Iso xs ys -> a m i xs -> FMap c m i ys
\end{minted}

The kind of the parameter \texttt{c} is not for the faint of heart:
\begin{center}
\texttt{c :: ((* -> *) -> * -> [*] -> *) -> Constraint} 
\end{center}
Luckily kinds are automatically inferred, therefore there is no need
to provide an explicit type signature for it\footnote{Unfortunately GHC 7.8 does not support kind synonyms.}.
Contrary to the simple example presented in \ref{subsec:ExtensibleFormat},
the parameters \texttt{m}, \texttt{i} and \texttt{xs} of the argument 
of \texttt{FMap} have been left explicit because the data type
itself is indexed over those.
Nevertheless the argument is not explicitly indexed by a constraint kind,
so that its kind is slightly simpler:

\begin{center}
\texttt{a :: (* -> *) -> * -> [ * ] -> *}
\end{center}

Following the pattern described in \ref{subsec:ExtensibleFormat},
a convenient type synonym is defined:
\begin{minted}{haskell}
type Use a c m i = c m i (a c)
\end{minted}

In order to retain the well-known syntax for functors, a smart constructor
is also provided:
\begin{minted}{haskell}
(<$>) :: Use a c m i => Iso args xs -> a c m i args -> FMap c m i xs
f <$> x = FMap f x
\end{minted}

	Analogous data types and relative smart constructors 
	have been defined for the other core combinators
	discussed in the previous sections, which include 
	\texttt{Applicative}, \texttt{Alternative} and \texttt{Monad} combinators.

	\paragraph{Semantic Functions}
	\label{par:SemanticsFun}
	The interpretation function \texttt{mkParser} has been converted into
	a type class constraint:
	
\begin{minted}{haskell}
class ParseWith (m :: * -> *) (i :: *) a where
  mkParser' :: a m i xs -> m (HList xs)
\end{minted}

	Also in this case \texttt{a} is not explicitly indexed with a constraint
	parameter, hence an appropriate entry point is needed
	for constraint-indexed types:

\begin{minted}{haskell}
mkParser :: Use a ParseWith m i => a ParseWith m i xs -> m (HList xs)
mkParser = mkParser'
\end{minted}

	Furthermore a completely general, yet correct, instance is given
	in terms of the underlying \texttt{Functor} instance:
\begin{minted}{haskell}
instance Functor m => ParseWith m i (FMap ParseWith) where
  mkParser' (FMap i f) = apply i <$> mkParser' f
\end{minted}
%$

	The transformation of the semantic function \texttt{mkPrinter}
	follows exactly the same pattern.
	Furthermore for the core combinators that belong to the
	\texttt{Applicative}, \texttt{Alternative} and \texttt{Monad} classes
	analogous general instances have been provided.
	As a result the only instances that must be manually added
	are those of \texttt{ParseSatisfy}	and \texttt{PrintToken}, 
	because they are library specific.
	To give a concrete example, the instance relative to
	\texttt{Parsec} \cite{Parsec} parser is given:

\begin{minted}{haskell}
instance Stream s m Char => ParseSatisfy (ParsecT s u m) Char where
  parseSatisfy = satisfy
\end{minted}

	\paragraph{Error Messages}
	Extending the format universe with new cases is straightforward.
	For example many parsing libraries include a combinator 
	that allows to provide an helpful error message,
	when a parser fails, 	typically which token was expected.

\begin{minted}{haskell}
<?> :: Parser a -> String -> Parser a
\end{minted}	

	The combinator is converted in a new data type and
	an appropriate smart constructor is defined:
\begin{minted}{haskell}
data Help c m i xs where
  Help :: c m i a => a m i xs -> String -> Help c m i xs

(<?>) :: Use a c m i => a c m i xs -> String -> Help c m i xs
f <?> msg = Help f msg
\end{minted}
	
	To keep the library as easily pluggable as possible an appropriate
	hook is defined, by means of another type class, following the
	example of the \texttt{ParseSatisfy} and \texttt{PrintToken} classes:

\begin{minted}{haskell}
class ParseHelp m where
  parseHelp :: m a -> String -> m a
  parseHelp = const
\end{minted}
	Furthermore in this case it is possible to provide an appropriate
	default behaviour, which simply ignores the given error message.
	For libraries that provides such combinators, like Parsec \cite{Parsec},
	the instance is straightforward:
	
\begin{minted}{haskell}
instance ParseHelp (ParsecT s u m) where
  parseHelp = (<?>)
\end{minted}

	Once more a generic instance of \texttt{ParseWith} is given,
	assuming \texttt{ParseHelp} in the context:
	
\begin{minted}{haskell}
instance ParseHelp m => ParseWith m i (Help ParseWith) where
  mkParser' (Help f msg) = parseHelp (mkParser' f) msg
\end{minted}
	
\section{Conclusion}
\label{sec:FormatConclusion}
\subsection{Discussion}
\label{subsec:Discussion}
The library presented in this chapter can effectively describe various real-world formats and derive automatically consistent parsers and printers 
for them.
Text-based and binary formats are both supported by abstracting
over the token and stream type.
The library relies on a minimal core of basic standard format combinators, that closely resemble well-known parser combinators and can be easily
inverted.
As a result the user of this library specifies a format just like writing a parser for it and receives the inverse printer for free.

\paragraph{Plug-in Framework}
The library Boomerang \cite{Boomerang}, currently available on Hackage,
provides similar support for invertible parsing and printing.
However, differently from the library proposed in this thesis, the parser
and printer backend are fixed and implemented from scratch.
Considering the abundance of sophisticated, efficient and mature 
parsing and printing libraries readily available it seems pointless to reinvent yet one more.
One of the benefits of this framework is that it allows to reuse any
existing library off the shelf, with minimal effort needed from the
final user. Arbitrary libraries can seamlessly be plugged in as long as they 
implement at least the \texttt{Alternative} type class, which is 
customary.

\paragraph{Extensible Framework}
The framework is extensible, letting the user tailor the library to his needs, by
introducing new primitives.
For example some parsing libraries provide specific functions
to increase the performance of certain operations, such as \texttt{skipMany}, or to selectively enable backtracking, as it happens with \texttt{try} in Parsec \cite{Parsec}.
The format representation discussed in \ref{subsec:FormatRevised} can be 
easily extended with new constructs, following the technique explained in
\ref{subsec:ExtensibleFormat}.

\paragraph{Alternative semantics}
The generic parsing and printing semantics given for the applicative combinators are inverse under few assumptions,
namely that \texttt{pure} does not consume any input when parsing and that \texttt{<*>} applies the second parser on the input left after the first parser is applied, which is the de-facto standard behaviour in parsing libraries.

Two opposite semantics are common for the choice 
combinator \texttt{<|>}: greedy and symmetric choice. 
Each library typically provides only one of them, 
depending on the specific parsing strategy implemented.
For instance Parsec, an industrial strenght parsing library \cite{Parsec}, 
provides a greedy choice operator, whose semantics is predictive parsing
with a look ahead of one token. On the other hand UU-Parsinglib, based on the work by Swiestra et al. \cite{UUParsingLib}, provides a symmetric choice operator that does not commit to any alternative. If more than one succeeds, then an ambiguous grammar is detected and a run-time failure occurs.

In the \texttt{Format} data type, the combinator \texttt{<|>} is not explicitly marked as greedy or symmetric choice, therefore it is up to the user to ensure that the corresponding printing operator has the appropriate inverse semantics.

% Arbitrary choice for apply/unapply in <$>
% It is worth pointing out that these operations are all type-safe.		

\subsection{Related and Future Work}
\label{subsec:FormatRelatedWork}
The problem of unifying parsing and printing in a single specification
has been studied extensively in literature.
Some work are based on arrows \cite{Alimarine, PolyDataConv}.
Alimarine et al.\ approach consists of writing invertible programs by construction, combining bidirectional arrows, i.e.\ BiArrows.
As an example of this technique they implement a parser using reversible
arrows, thus getting the corresponding printer for free.
However this technique stands back from the fairly ordinary 
applicative or monadic style, so common in parsing libraries.
In this library formats are described using these established styles, hence it is more likely to be adopted.
Boespflug proposes a similar embedded DSL based on reversible combinators called cassettes and that relies on continuation passing style and rank 2 types 
for composition \cite{Boespflug}.
The translation to direct style reveals that the continuations are impure
and make use of control effects. It is particularly cumbersome and unnatural to define cassette for user-defined data type: code involving continuation 
passing style is notoriously hard to understand, maintain and debug.
Matsuda and Wang develop FliPpr, a quite involved programming transformation system that inverts the specification of a pretty printer and 
produces the corresponding parser \cite{FliPpr}.
Similarly to the library developed in this thesis, the system does not
reimplement parser and printer from scratch, but rather 
reuses existing libraries as backend. In addition their technique
provides a fine-grained control over the pretty-printing, which 
however has the drawback of somewhat cluttering the input grammar 
with pretty printing annotations.
For example the \emph{biased choice} operator \texttt{<+} separates 
pretty from ugly patterns, which are nevertheless accepted when
parsing. Lastly it is arguably more natural and desirable 
to implement a parser and get the corresponding printer for free, rather
than the other way around.
The library proposed in this thesis does not aim to print pretty output directly,
however it is still possible to choose one of the existing pretty printing
library \cite{Hughes95thedesign, Wadler98aprettier, SwierstraPP} as backend, for this purpose. Printers for certain formats are inherently ambiguous and
result in non-termination, such as the combination of the primitive version
of \texttt{many} and \texttt{some} with \emph{trivial} formats \ref{subsec:Many}. In these circumstances the function \texttt{atMost} 
tunes a printer and selects a \emph{pretty} version.

There is also a relevant series of works on type-safe variants of 
C \texttt{printf} and \texttt{scanf} formatting functions.
Danvy's approach consists of an embedded DSL, that exploits continuation
passing style \cite{Danvy}.
Asai reflects on his work and reveals that it is based on delimited continuations \cite{Asai}. He then elaborates three new solutions, including
two in direct style.
Hinze employs functor type-indexed formats, which are combined to
compute the types of the expected arguments.
The library developed in this thesis is more expressive than 
common string formats, because it supports user-defined data types and
recursive formats. 
The format representation proposed here can be effectively employed to implement a type safe version \texttt{printf} and \texttt{scanf}.
However the arguments to the printer are actually collected in a single heterogeneous list: it would be interesting to investigate whether it is possible
to provide the \texttt{printf} variadic interface, exploiting
Haskell advanced type features. While the type of the function
can be easily computed from the index of the format using a closed type family, it seems highly non-trivial to curry a function that takes a list-index
data type as argument. I conjecture that a continuation-passing style solution
could solve the problem.

The library presented in this chapter is greatly inspired by the
work on Invertible Syntax Descriptions by Rendel and Ostermann \cite{Rendel10ISD}.
Following the example of parsing libraries based on applicative functors, 
they define a small core of invertible primitives,
which can be combined to build complex invertible syntax descriptions.
Their combinators are defined as methods of ad-hoc classes 
such as \texttt{IsoFunctor} and \texttt{IsoApplicative} that
resemble the standard Applicative and Functor classes, but
are tailored for partial isomorphism.
Instead this library, in its simplest description \ref{sec:Format}, 
represents basic formats as a universe.
The parsing and printing semantics are obtained as an \emph{interpretation}, an approach similar to that proposed by Swiestra \cite{PowerOfPi}.

\paragraph{Partial Isomorphism}
Both the library implemented in this thesis and that of Rendel et al.\ 
rely on partial isomorphisms to couple a function and its inverse in a single entity, however they are slightly different.
Their isomorphism is symmetrical, because the two functions are both
partial. As a result the corresponding algebra of partial isomorphism 
is slightly more expressive. For example a partial isomorphism
for \texttt{foldl} is derived from a minimal set of primitive
isomorphisms, as a small-step abstract machine.
The partial isomorphism used in this library on the other hand
is partial only in the inverse function.
Its asymmetry precludes the basic combinator \texttt{inverse}, that
is essential to derive \texttt{foldl}, which is then implemented as a primitive. However, regardless of which partial isomorphism is used, 
\texttt{foldr} could not be derived following the same approach.
Since \texttt{foldr} is strictly more expressive than \texttt{foldl} it is an
open question whether the algebra of partial isomorphism is expressive
enough to include it.
It is also worth pointing out that \texttt{foldr} can be inverted only under certain specific conditions.
\begin{center}
\texttt{unfoldr g (foldr f z xs) ≡ xs}
\end{center}
Specifically:
\begin{center}
\begin{tabular}{cccc}
\texttt{g}  &\texttt{(f x y)} &\texttt{=} &\texttt{Just (x, y)} \\
\texttt{g}  &\texttt{z}        &\texttt{=} &\texttt{Nothing}
\end{tabular}
\end{center}

The two isomorphisms differ also in the kind of their parameters:
list of types (\texttt{[*]}) are used in this library, while
Randel employs simple types (\texttt{*}).
At first sight these representations might look equivalent, because
the unit type \texttt{()} corresponds to the empty list \texttt{'[]},
and nested pairs can be used to collect together heterogeneous types
just like type level lists.
However it turns out that the type level representation is more precise
and therefore more flexible.
Consider for example the two alternative signatures of the applicative operator \texttt{<*>}:

\begin{minted}{haskell}
<*> :: f a -> f b -> f (a, b)
<*> :: Format f xs -> Format f ys -> Format f (xs :++: ys)
\end{minted}

Combining two \emph{trivial} formats \texttt{p} and \texttt{q} with 
\texttt{<*>} should result in another trivial format, however this happens
only with the more accurate list representation:
\begin{minted}{haskell}
p <*> q :: f ((), ())
p <*> q :: Format f '[]
\end{minted}

The interaction of trivial formats with list based combinators such as \texttt{many} and \texttt{some} shows the same wicked behaviour:

\begin{minted}{haskell}
many p :: f [()]
many p :: Format f '[]
\end{minted}

The strict kind distinction between target types and
container allows to precisely manipulate the former using closed
type families such as \texttt{Map} and \texttt{:++:}, making the
combinators more composable. Furthermore several
desired properties of the partial isomorphism algebra, such as associativity, follow directly from the properties of the container type.

\paragraph{Design}
The technique presented by Rendel et al.\ is demonstrated 
on a proof-of-concept parsing and pringing library.
The instances given for the newly created type classes,
require full access to the library, however
it is customary in libraries not to expose fundamental data types, 
in order to prevent users from breaking internal invariants.
The library implemented in this thesis
does not rely on a such privileged view, on the contrary, 
since the behaviour of the core combinators is virtually
standard, completely general and reusable instances are given.
Furthermore Rendel's library could support generically only
monadic parsers, because of the symmetric partiality of
the functions stored in his isomorphism.
For example consider the \texttt{IsoFunctor} combinator
\texttt{<\$> :: Iso a b -> f a -> f b} and the standard
\texttt{Functor} combinator \texttt{<\$> :: (a -> b) -> f a -> f b}.
In the latter the first argument is a pure function and it is impossible
to appropriately compose it with the partial function of
the isomorphism, without the increased
expressive power of the monadic binding.
Concretely the only admissible generic instance that can be 
given for \texttt{IsoFunctor} is:
\begin{minted}{haskell}
instance Monad f => IsoFunctor f where 
  i <$> p = do
    x <- p
    case apply i of
      Just y -> return y
      Nothing -> fail ""
\end{minted} 
% $
On the other hand, the generic instance given in \ref{par:SemanticsFun} for
the functor \texttt{FMap} accordingly requires its underlying semantics to be
only a \texttt{Functor}.
Furthermore using the symmetric partial isomorphism they implement their 
own \texttt{satisfy} function, named \texttt{subset :: (a -> Bool) -> Iso a a},
and instead assume in the class \texttt{Syntax} a method
\texttt{token} that returns the next token in the stream.
This design choice does not cope well with existing parsing libraries,
which usually provide \texttt{satisfy} as a primitive and implement
token as \texttt{satisfy (const True)}. An unfortunate consequence 
of this mismatch is that a generic instance would poorly interact with
the underlying library.
In this library the isomorphism \texttt{subset} cannot be defined because
the \texttt{apply} function is not partial, but more importantly it
is not needed, because it is assumed as primitive in \texttt{ParseSatisfy}.

Rendel et al.\ recognize the problem of providing a suitable interface,
in order to support existing libraries and suggest subclassing
as the main extension mechanism. This is not an option
for our library, because it is not based on type classes, however
extensibility is equivalently achieved, by means of a novel
solution to the expression problem \ref{subsec:ExtensibleFormat}.

\paragraph{Monadic Formats}
In section \ref{subsec:Monadic} the format universe is extended 
with a monadic bind operator. 
Its signature is not as general as the conventional one for practical purposes.
As far as I know this is the first library that provides
invertible parser and printers for context sensitive grammars, even
though with some limitations.
It would be interesting to investigate further this topic, in order
to give to this combinator the standard degree of generality:
\begin{minted}{haskell}
(>>=) :: Format f xs -> (HList xs -> Format f ys) -> Format f ys
\end{minted}
The problematic part lies in the printer semantics, in which the
dependency embedded by the continuation has to be somehow inverted. Specifically, in order to correctly invert the parser, 
from an arbitrary \texttt{HList ys} the corresponding \texttt{HList xs}
has to be retrieved.
The invertible programming paradigm could provide some insight on this problem, though it is important to keep in mind that only injective functions
can be inverted, consequently this approach could restrict the 
class of grammar accepted.
On the other hand the solution proposed in the library is conservative, but sound, because no limitation whatsoever is imposed.

\paragraph{Formal Semantics}
In this thesis the correct invertibility of the formats provided by this
library has not been formally proved.
It would be interesting to investigate its formal properties, using
a proof assistant with dependent types.
The work on Total Parser Combinator by Danielsson 
\cite{TotalParserCombinator} is particularly relevant and could
represent a suitable base for this line of research.
Exploiting dependent types and mixing induction and coinduction, 
he develops a monadic parser combinators library that guarantees 
termination and supports left recursion.
The same techniques are then employed to study correct-by-construction
pretty printers \cite{CorrectByConstruction}. The main result of the paper
is the proof that those printers satisfy the round-trip property,
i.e.\ if a value is pretty printed and the resulting string parsed with respect
to the same grammar, the original value is obtained.
Future work should strive to prove the same result for this format library.
On the other hand I expect a weaker property for the converse theorem, in fact 
the ambiguity problem of the printers discussed in \ref{par:Termination}
is resolved arbitrarily. Further research should study a suitable relation
\texttt{s₁ ⊆ s₂}, which denotes that the representation \texttt{s₂}
is as ``pretty'' as \texttt{s₁}.
The desired theorem would then be \texttt{print (parse s) ⊆ s}.
Analyzing double semantics specification in a proof assistant would
also require to precisely describe the inductive and coinductive definitions
of the format combinators, hence explaining which formats can be 
correctly defined in this framework.

% TODO remove
%	\bibliographystyle{plain}
% 	\bibliography{../svc}

\end{document}
