\documentclass[../Thesis.tex]{subfiles}

\begin{document}

\chapter{Conclusion}
\label{chp:Conclusion}
In this thesis I have analyzed the problem of structure-aware revision control,
which aims to improve the quality of version control systems, by
exploiting the knowledge of how data is encoded in a file.

Firstly I have developed an EDSL for binary and text-based data formats, which
automatically derives inverse parser and printer, given a format description.
Since a unique specification is given for each
format, this technique ensures that parsers and printers are 
always synchronized, therefore ensuring round-trip behaviour.
A format description allows a version control system 
to access to the data contained in a file and its structure, so that 
it can detect changes more precisely. In addition
after a merge, the data can be serialized back to a file according to its format.

Secondly I have implemented a data-type generic \texttt{diff} and \texttt{diff3}
algorithm in Haskell. The data stored in a file is parsed and represented as
a data type, which can be easily converted to the heterogeneous rose trees
data type, a generic representation which is used in the \texttt{diff} and 
\texttt{diff3} algorithms. The algorithms
have been employed in a proof-of-concept structure-aware version
control system that shows the applicability of the theories studied in this 
thesis.

Lastly I have developed a formal model in the Agda proof assistant, 
with which I have studied the two algorithms and their formal properties.
The model provides an unambiguous specification of the algorithms
and accurately describe their semantics.
In addition the properties proved are a precious 
source of information, useful to interpret and further 
describe their behaviour. The formal model gives the foundations
to further investigate the semantics of structure-aware version control
systems.

The contributions of this thesis can be summarized in:
\begin{itemize}
	\item An EDSL for describing data formats that unifies parsing and printing.
	\item A data type generic \texttt{diff} and \texttt{diff3} algorithm.
	\item A formal model that describes the semantics of these algorithms.
	\item A proof-of-concept structure-aware version control system.
\end{itemize}

\end{document}